% !TEX root = ../document.tex

\newcommand{\R}{\mathds{R}}
\newcommand{\Z}{\mathds{Z}}
\newcommand{\N}{\mathds{N}}
\newcommand{\Q}{\mathds{Q}}
\newcommand{\K}{\mathds{K}}
\newcommand{\C}{\mathds{C}}
\newcommand{\B}{\mathds{B}}
\newcommand{\F}{\mathds{F}}
\newcommand{\p}{\mathfrak{p}}
\newcommand{\Pot}{\mathcal{P}}
\newcommand{\id}{\textup{id}}
\newcommand{\Ker}{\textup{Ker}}
\newcommand{\Image}{\textup{Im}}
\newcommand{\la}{\langle}
\newcommand{\ra}{\rangle}
\newcommand{\gdw}{\Leftrightarrow}

\newtheorem{lemma}{Lemma}[subsection]
\newtheorem{prop}[lemma]{Proposition}
\newtheorem{defprop}[lemma]{Definition and Proposition}
\newtheorem{satz}[lemma]{Satz}
\newtheorem{thm}[lemma]{Theorem} 
\newtheorem{kor}[lemma]{Korollar} 
\newtheorem{folg}[lemma]{Folgerung}
\newenvironment{bew}{\begin{proof}[Beweis]}{\end{proof}}

\theoremstyle{definition} 
\newtheorem{def1}[lemma]{Definition} 
\newtheorem{bem}[lemma]{Bemerkung}
\newtheorem{bsp}[lemma]{Beispiel}
\newtheorem{notation}[lemma]{Notation}
\newtheorem{algorithm}[lemma]{Algorithmus}

\newcommand{\uf}[1]{\frac{1}{#1}} % unit fraction \frac{1}{n}

%for the \usepackage[math]{cellspace}
\cellspacetoplimit 2pt
\cellspacebottomlimit 2pt