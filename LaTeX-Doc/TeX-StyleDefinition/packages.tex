% !TEX root = ../document.tex

% Anpassung an Landessprache ---------------------------------------------------
\usepackage{babel}

% Umlaute ----------------------------------------------------------------------
%   Umlaute/Sonderzeichen wie äüöß direkt im Quelltext verwenden (CodePage).
%   Erlaubt automatische Trennung von Worten mit Umlauten.
% ------------------------------------------------------------------------------
\usepackage[T1]{fontenc}
\usepackage{textcomp} % Euro-Zeichen etc.

% Schrift ----------------------------------------------------------------------
\usepackage{lmodern} % bessere Fonts
\usepackage{relsize} % Schriftgröße relativ festlegen
\usepackage{dsfont}

% Tabellen ---------------------------------------------------------------------
\PassOptionsToPackage{table}{xcolor}
\usepackage{tabularx}
% für lange Tabellen
\usepackage{longtable}
\usepackage{array}
\usepackage{ragged2e}
\usepackage{lscape}
\newcolumntype{w}[1]{>{\raggedleft\hspace{0pt}}p{#1}} % Spaltendefinition rechtsbündig mit definierter Breite

% Grafiken ---------------------------------------------------------------------
\usepackage[dvips,final]{graphicx} % Einbinden von JPG-Grafiken ermöglichen
\usepackage{graphics} % keepaspectratio
\usepackage{wrapfig}
\usepackage{floatflt} % zum Umfließen von Bildern
\usepackage{float}
\usepackage{subcaption}
\graphicspath{{Bilder/}} % hier liegen die Bilder des Dokuments

% Fußnoten ---------------------------------------------------------------------
\let\oldfootnote\footnote
\renewcommand\footnote[1]{\oldfootnote{\hspace{2mm}#1}}

% Sonstiges --------------------------------------------------------------------
\usepackage[titles]{tocloft} % Inhaltsverzeichnis DIN 5008 gerecht einrücken
\usepackage{amsmath,amsfonts,amssymb,amsthm} % Befehle aus AMSTeX für mathematische Symbole
\usepackage{lipsum}
\usepackage{enumitem} % anpassbare Enumerates/Itemizes
\usepackage{xspace} % sorgt dafür, dass Leerzeichen hinter parameterlosen Makros nicht als Makroendezeichen interpretiert werden

\usepackage{makeidx} % für Index-Ausgabe mit \printindex
\usepackage[printonlyused]{acronym} % es werden nur benutzte Definitionen aufgelistet

% Einfache Definition der Zeilenabstände und Seitenränder etc.
\usepackage{setspace}
\usepackage[margin = 2in]{geometry}

% Symbolverzeichnis
\usepackage[intoc]{nomencl}
\let\abbrev\nomenclature
\renewcommand{\nomname}{Abkürzungsverzeichnis}
\setlength{\nomlabelwidth}{.25\hsize}
\renewcommand{\nomlabel}[1]{#1 \dotfill}
\setlength{\nomitemsep}{-\parsep}

\usepackage{varioref} % Elegantere Verweise. „auf der nächsten Seite“
\usepackage{url} % URL verlinken, lange URLs umbrechen etc.

\usepackage{chngcntr} % fortlaufendes Durchnummerieren der Fußnoten
% \usepackage[perpage]{footmisc} % Alternative: Nummerierung der Fußnoten auf jeder Seite neu

\usepackage{ifthen} % bei der Definition eigener Befehle benötigt
\usepackage{todonotes} % definiert u.a. die Befehle \todo und \listoftodos
\usepackage[square]{natbib} % wichtig für korrekte Zitierweise

% PDF-Optionen -----------------------------------------------------------------
\usepackage{pdfpages}
\pdfminorversion=5 % erlaubt das Einfügen von pdf-Dateien bis Version 1.7, ohne eine Fehlermeldung zu werfen (keine Garantie für fehlerfreies Einbetten!)
\usepackage[
    bookmarks,
    bookmarksnumbered,
    bookmarksopen=true,
    bookmarksopenlevel=1,
    colorlinks=true,
% diese Farbdefinitionen zeichnen Links im PDF farblich aus
%    linkcolor=AOBlau, % einfache interne Verknüpfungen
%    anchorcolor=AOBlau,% Ankertext
%    citecolor=AOBlau, % Verweise auf Literaturverzeichniseinträge im Text
%    filecolor=AOBlau, % Verknüpfungen, die lokale Dateien öffnen
%    menucolor=AOBlau, % Acrobat-Menüpunkte
%    urlcolor=AOBlau,
% diese Farbdefinitionen sollten für den Druck verwendet werden (alles schwarz)
    linkcolor=black, % einfache interne Verknüpfungen
    anchorcolor=black, % Ankertext
    citecolor=black, % Verweise auf Literaturverzeichniseinträge im Text
    filecolor=black, % Verknüpfungen, die lokale Dateien öffnen
    menucolor=black, % Acrobat-Menüpunkte
    urlcolor=black,
%
    %backref, % Quellen werden zurück auf ihre Zitate verlinkt
    pdftex,
    plainpages=false, % zur korrekten Erstellung der Bookmarks
    pdfpagelabels=true, % zur korrekten Erstellung der Bookmarks
    hypertexnames=false, % zur korrekten Erstellung der Bookmarks
    linktocpage % Seitenzahlen anstatt Text im Inhaltsverzeichnis verlinken
]{hyperref}
% Befehle, die Umlaute ausgeben, führen zu Fehlern, wenn sie hyperref als Optionen übergeben werden
\hypersetup{
    pdftitle={\titel -- \untertitel},
    pdfauthor={\autorName},
    pdfcreator={\autorName},
    pdfsubject={\titel -- \untertitel},
    pdfkeywords={\titel -- \untertitel},
}


% zum Einbinden von Programmcode -----------------------------------------------
\usepackage{listings}
\usepackage[]{xcolor}

%\definecolor{hellgelb}{rgb}{1,1,0.9}
%\definecolor{colKeys}{rgb}{0,0,1}
%\definecolor{colIdentifier}{rgb}{0,0,0}
%\definecolor{colComments}{rgb}{0,0.5,0}
%\definecolor{colString}{rgb}{1,0,0}
%\lstset{
%    float=hbp,
%	basicstyle=\footnotesize,
%    identifierstyle=\color{colIdentifier},
%    keywordstyle=\color{colKeys},
%    stringstyle=\color{colString},
%    commentstyle=\color{colComments},
%    backgroundcolor=\color{hellgelb},
%    columns=flexible,
%    tabsize=2,
%    frame=single,
%    extendedchars=true,
%    showspaces=false,
%    showstringspaces=false,
%    numbers=left,
%    numberstyle=\tiny,
%    breaklines=true,
%    breakautoindent=true,
%	captionpos=b,
%}
%\lstdefinelanguage{cs}{
%	sensitive=false,
%	morecomment=[l]{//},
%	morecomment=[s]{/*}{*/},
%	morestring=[b]",
%	morekeywords={
%		abstract,event,new,struct,as,explicit,null,switch
%		base,extern,object,this,bool,false,operator,throw,
%		break,finally,out,true,byte,fixed,override,try,
%		case,float,params,typeof,catch,for,private,uint,
%		char,foreach,protected,ulong,checked,goto,public,unchecked,
%		class,if,readonly,unsafe,const,implicit,ref,ushort,
%		continue,in,return,using,decimal,int,sbyte,virtual,
%		default,interface,sealed,volatile,delegate,internal,short,void,
%		do,is,sizeof,while,double,lock,stackalloc,
%		else,long,static,enum,namespace,string},
%}
%\lstdefinelanguage{natural}{
%	sensitive=false,
%	morecomment=[l]{/*},
%	morestring=[b]",
%	morestring=[b]',
%	alsodigit={-,*},
%	morekeywords={
%		DEFINE,DATA,LOCAL,END-DEFINE,WRITE,CALLNAT,PARAMETER,USING,
%		IF,NOT,END-IF,ON,*ERROR-NR,ERROR,END-ERROR,ESCAPE,ROUTINE,
%		PERFORM,SUBROUTINE,END-SUBROUTINE,CONST,END-FOR,END,FOR,RESIZE,
%		ARRAY,TO,BY,VALUE,RESET,COMPRESS,INTO,EQ},
%}
%\lstdefinelanguage{php}{
%	sensitive=false,
%	morecomment=[l]{/*},
%	morestring=[b]",
%	morestring=[b]',
%	alsodigit={-,*},
%	morekeywords={
%		abstract,and,array,as,break,case,catch,cfunction,class,clone,const,
%		continue,declare,default,do,else,elseif,enddeclare,endfor,endforeach,
%		endif,endswitch,endwhile,extends,final,for,foreach,function,global,
%		goto,if,implements,interface,instanceof,namespace,new,old_function,or,
%		private,protected,public,static,switch,throw,try,use,var,while,xor
%		die,echo,empty,exit,eval,include,include_once,isset,list,require,
%		require_once,return,print,unset},
%}

% Glossar -------------------------------------------------------------------------
\usepackage[toc,section,style=altlist,nonumberlist]{glossaries}
\makeglossaries
