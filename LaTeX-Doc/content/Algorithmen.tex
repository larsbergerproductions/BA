\section{Algorithmen zur Erstellung Ägyptischer Brüche}
Im Folgenden sollen verschiedene Algorithmen erklärt und verglichen werden, mit welchen sich rationale Brüche in solche ägyptischer Schreibweise zerlegen lassen. Die dabei aufgezeigten Methoden wurden über mehrere Jahrhunderte hinweg entwickelt und weisen dementsprechend signifikante Unterschiede auf. Im Anschluss an die Erklärung der Algorithmen soll ein Vergleich gezogen werden, der die Effizienz anhand der Kriterien Anzahl der verwendeten Stammbrüche und Länge des größten Nenners vergleicht.
Die nun betrachteten Algorithmen werden sein:
\begin{itemize}
	\item der Fibonacci-Sylvester-Algorithmus (auch: Greedy-Algorithmus)
	\item der Farey-Folgen-Algorithmus
	\item der Kettenbruch-Algorithmus
\end{itemize}