% !TEX root = ../document.tex
\section{Anhang}\label{sec:code}

Zur Berechnung der Testreihen aus Abschnitt \ref{sec:Testreihen} wurde ausschließlich der nachfolgend aufgeführte, selbst entwickelte Code in PARI/GP, Version \emph{2.9.4} umgesetzt.

\subsection{Hilfsfunktionen}
Alle genutzten, nicht in PARI/GP enthaltenen Funktionen sind im Folgenden aufgelistet.

\emph{listsum(list)} berechnet die Summe eines Verbunddatentyps, \bspw einer Liste oder eines Vektors. \emph{listmin(list)} berechnet das Minimum einer Liste, \emph{listmax(list)} das Maximum.
\lstinputlisting[language=gp, firstline=1, lastline=3]{../PARI/algorithms.gp}

\emph{contains(list, element)} durchsucht einen Verbunddatentyp und gibt 1 zurück, falls \emph{element} in \emph{list} enthalten ist, 0 sonst.
\lstinputlisting[language=gp, firstline=90, lastline=97]{../PARI/algorithms.gp}

\emph{reversevecsort(vect)} sortiert \emph{vect} in absteigender Reihenfolge.
\lstinputlisting[language=gp, firstline=99, lastline=105]{../PARI/algorithms.gp}

\emph{FareySeries(order)} berechnet die Farey-Folge der Ordnung \emph{order} und gibt diese als Vektor zurück.
\lstinputlisting[language=gp, firstline=107, lastline=117]{../PARI/algorithms.gp}

\emph{findAdjacent(Fs, fraction)} sucht in der Farey-Folge \emph{Fs} den adjazenten Bruch zu \emph{fraction}, der kleiner ist. Diese Funktion wird seit der Implementierung von \emph{findAdjacentFrel} nicht mehr verwendet und dient seither nur dem Vergleich.
\lstinputlisting[language=gp, firstline=119, lastline=138]{../PARI/algorithms.gp}

\emph{findAdjacentFrel(fract)} berechnet den kleineren der zu \emph{fract} adjazenten Brüche mithilfe des relevanten Teils der Farey-Fogle, wie in Beispiel \ref{bsp:Frel} gezeigt und stellt damit die wesentlich effizientere Alternative zu \emph{findAdjacent(Fs,fraction)} dar.
\lstinputlisting[language=gp, firstline=142, lastline=156]{../PARI/algorithms.gp}

\emph{mediant(frac1, frac2)} berechnet die Mediante zweier Brüche nach Definition \ref{def:mediant}.
\lstinputlisting[language=gp, firstline=140, lastline=140]{../PARI/algorithms.gp}

\emph{findDivisorsOf\_k\_addingup\_n(k,n)} sucht jene Teiler von $k$ heraus, die in ihrer Summe $n$ ergeben.
\lstinputlisting[language=gp, firstline=181, lastline=192]{../PARI/algorithms.gp}

\emph{printEgypFrac(arguments)} nimmt eine Liste mit Argumenten entgegen und gibt die String-Konkatenation dieser zurück.
\lstinputlisting[language=gp, firstline=217, lastline=234]{../PARI/algorithms.gp}


\subsection{PARI/GP Code für den Greedy-Algorithmus}\label{code:greedy}
Die Funktion \emph{fibonacci\_sylvester(fraction, stepsize, start)} berechnet mittels des gleichnamigen Algorithmus die entsprechende Ägyptische Darstellung des Bruchs \emph{fraction}. Das Argument \emph{stepsize} gibt an, um wie viel der Nenner eines Kandidaten bei Bedarf mindestens erhöht wird. Der zuerst untersuchte Kandidat ist $\uf{\text{start}}$. Damit wird die Funktionalität zur Verfügung gestellt, \bspw nur nach Stammbrüchen mit geraden oder ungeraden Nennern zu suchen.\\
Letztendlich ruft der Nutzer aber nur die Stellvertreterfunktionen \emph{greedy}, \emph{greedy\_odd} oder \emph{greedy\_even} auf, die die Hauptfunktion \emph{fibonacci\_sylvester} mit Standardwerten für \emph{stepsize} und \emph{start} nutzen.
\lstinputlisting[language=gp, firstline=5, lastline=39]{../PARI/algorithms.gp}


\subsubsection{Optimierter Greedy-Algorithmus}\label{code:greedy_fast}
Zur Effizienzsteigerung wurde der Greedy-Algorithmus nochmals mit einer wesentlich effizienteren Suche der Nenner umgesetzt, die sich mit dem Namen ''Double \& Add'' beschreiben lässt und dem Prinzip der Ägyptischen Multiplikation aus Abschnitt \ref{subsec:egypMult} entspricht. Die Funktionalität der Suche des größten Stammbruchs $max\{\uf{n}: n \in \N\}$, der kleiner als ein gegebener rationaler Bruch ist, wurde in die Funktion \emph{largestUnitFractionLEQ} ausgelagert. Der Rest des Algorithmus läuft ab wie im Anhang \ref{code:greedy} beschrieben.
\lstinputlisting[language=gp, firstline=41, lastline=88]{../PARI/algorithms.gp}


\subsection{PARI/GP Code für den Farey-Folgen-Algorithmus}\label{code:fareyseries}
Die Implementierung des Farey-Folgen-Algorithmus greift auf eine Dauerschleife zurück, aus der ausgebrochen wird, sobald der Nenner des betrachteten adjazenten Bruchs $1$ ist, was laut Algorithmus \ref{algo:FareySeries} das Abbruchkriterium ist. \emph{adjacent} ist dem Namen entsprechend der aktuelle, kleinere adjazente Bruch zum aktuell untersuchten Bruch \emph{current\_fraction}, \emph{remainder} ist $\uf{qs}$ aus demselben Algorithmus. Wie üblich stellt \emph{result} die Liste der Ergebnissummanden dar.
\lstinputlisting[language=gp, firstline=159, lastline=179]{../PARI/algorithms.gp}

\subsection{PARI/GP Code für den Binäralgorithmus}
\emph{p} und \emph{q} entsprechen Zähler und Nenner des Bruchs \emph{fraction}; \emph{r} und \emph{s} sind die natürlichen Zahlen, aus denen sich gemäß Algorithmus \ref{algo:binary} $qs+r = pN_k$ ergibt. Die Liste \emph{summands} enthält die Summanden der Zweierpotenzen, die in Summe den Zähler des aktuell betrachteten Bruchs ergeben. Das Ergebnis ist die Liste der Summanden des Ägyptischen Bruchs, die in \emph{result} gespeichert wird.
\lstinputlisting[language=gp, firstline=194, lastline=214]{../PARI/algorithms.gp}


\subsection{Beispielhafter Testreihenaufruf}
Zur Reihentestung der Algorithmen wurde jeweils eine eigene Funktion geschrieben, die entsprechende Zähler und Nenner bestimmte, den Algorithmus damit aufrief, das Ergebnis auswertete und die Auswertung in eine CSV\footnote{comma-separated values}-Datei schrieb. Die ist hier am Beispiel für den Binär-Algorithmus gezeigt, da diese Funktion für alle Algorithmen größtenteils gleich ist. Die Parameter \emph{start} und \emph{end} geben die untere \bzw obere Schranke für den Nenner an und definiere somit den Testbereich.
\lstinputlisting[language=gp, firstline=352, lastline=373]{../PARI/algorithms.gp}