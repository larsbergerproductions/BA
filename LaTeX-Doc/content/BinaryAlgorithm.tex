\subsection{Binäralgorithmus}

Dass die Ägypter schon damals implizit eine Art Binärschreibweise für ihre Multiplikation und Division verwendeten, wie in Abschnitt \ref{sec:arithmetic} erläutert, ist inzwischen bekannt. Nun kann man den gleichen Umstand auch für einen Konstruktionsalgorithmus verwenden, wie im Folgenden gezeigt wird.\\
Seien $m, n \in \N$ und $N = 2^n$. Alle $m < N$ lassen sich als Summe paarweise verschiedener Teiler von $N$ beschreiben, folglich also mit maximal $n$ Summanden. Praktisch lässt sich dies am Einfachsten anhand der Binärdarstellung von $m$ erkennen.

\begin{algorithm}\label{algo:binary}
	Sei $\frac{p}{q} \in Q_+, \, \frac{p}{q} < 1$ in gekürzter Form und $k \in \N$.
	\begin{enumerate}
		\item Finde $N_{k-1} < q \leq N_k$ wobei $N_k=2^k$ ist.
		\item Falls $q = N_k$, schreibe $p$ als Summe von Teilern von $N_k$, hier $d_i$ genannt:
		$$\frac{p}{q} = \sum_{i=1}^{j} \frac{d_i}{N_k}=  \sum_{i = 1}^{j}\frac{1}{\frac{N_k}{d_i}}$$
		\item Sonst seien $s, r \in \N, 0 \leq r < q$ so gewählt, dass:
		$$pN_k = qs+r.$$
		Es folgt:
		$$\frac{p}{q} = \frac{p N_k}{q N_k} = \frac{qs + r}{q N_k} = \frac{s}{N_k} + \frac{r}{q N_k}.$$
		\item Schreibe $s = \sum d_i$ und $r = \sum d_i'$, wobei $d_i, d_i'$ jeweils paarweise verschiedene Teiler von $N_k$ sind.
		\item Erhalte den Ägyptischen Bruch:
		$$\sum \uf{\frac{N_k}{d_i}} + \sum \uf{\frac{q N_k}{d_i'}}$$
	\end{enumerate}
\end{algorithm}

\begin{satz}
	Der Binäralgorithmus, wie in \ref{algo:binary} beschrieben, terminiert für jede Eingabe.
\end{satz}

\begin{bew}
	Falls $q = N_k$, hat das Ergebnis offensichtlich maximal $k$ Terme, da sich $N_k$ als Summe seiner $\log_2 N = k$ Terme schreiben lässt; in der Summe sind die $d_i$ paarweise verschieden, somit auch die $\frac{N_k}{d_i}$ und es gibt keinen Term mehrfach.\\
	Falls $q < N_k$, gilt
	$$qs+r = p N_k < q N_k.$$
	Somit gibt es eine Zerlegung in Ägyptische Brüche jeweils für s und r. Diese beiden Zerlegungen liefern für sich genommen nach dem Argument aus Fall ''$q = N_k$'' paarweise verschiedene Stammbrüche. Dass diese sogar zusammengenommen paarweise verschieden sind, folgt daraus, dass die zu $s$ gehörenden Nenner immer Zweierpotenzen sind, die zu $r$ gehörigen aber niemals.
\end{bew}


\paragraph{Abschätzung}Sei $\frac{p}{q} \in \Q_+, \frac{p}{q} < 1$. Gong zeigte, dass die Komplexitätsklasse für die Anzahl der Terme $O(\log n)$ ist \cite[S. 12]{Gong1992}. \\
Falls $q = N_k$ ist, folgt, dass der maximale Nenner $q$ ist. Anderenfalls definiert sich dieser aus dem Term $\frac{r}{qN_k}$ mit einer $1$ in der Zerlegung von $r$ in Teiler von $N_k$. Schlimmstenfalls ist $q$ dann der Nachfolger einer Zweierpotenz und es gilt 
$$N_k = 2(q-1).$$
Daraus folgt als kleinstmöglicher Bruch $\uf{q \cdot 2(q-1)} = \uf{2(q^2-q)}$, was die obere Schranke für den größten Nenner $2(q^2-q)$ impliziert, die in der Komplexitätsklasse $O(q^2)$ enthalten ist.

Da es für diesen Algorithmus zwei Fälle gibt, soll für jeden Fall ein Beispiel gezeigt werden. Dafür beginnen wir mit dem einfachen Fall.

\begin{bsp}
	Sei $\frac{9}{16}$ der zu zerlegende Bruch. $N_k = 16$, da $8<16\leq16$.\\
	Da $16 = 16$, wird nach Schritt (2) aus Algorithmus \ref{algo:binary} verfahren:
	$$\frac{9}{16} = \frac{8+1}{16} = \uf{2}+\uf{16}.$$
\end{bsp}
Ist der Nenner des zu zerlegenden Bruchs also eine Zweierpotenz, terminiert der Algorithmus sehr schnell. Anders ist dies, sollte es sich um keine Zweierpotenz handeln, wie das nächste Beispiel zeigt.
\begin{bsp}
	Sei $\frac{21}{23}$ der zu zerlegende Bruch. $N_k = 32$, da $16 < 23 \leq 32$.\\
	Da $23<32$ ist, wird nach Schritt (3) aus Algorithmus \ref{algo:binary} verfahren:
	$$\frac{21}{23} = \frac{21 \cdot 32}{23 \cdot 32}.$$
	Aus der Bedingung $0<r<32$ folgt $s = 29$ und $r = 5$, da $$qs+r = 29 \cdot 23 + 5 = 21 \cdot 32 = p N_k.$$
	Daraus folgt:
	$$\frac{21}{23} = \frac{23 \cdot 29+5}{23 \cdot 32} = \frac{29}{32} + \frac{5}{23 \cdot 32}.$$
	Aus $$\frac{29}{32} = \frac{16+8+4+1}{32} = \uf{2} + \uf{4} + \uf{8} + \uf{32}$$ und $$\frac{5}{23 \cdot 32} = \frac{4+1}{23 \cdot 32} = \uf{23 \cdot 8} + \uf{23 \cdot 32}$$ folgt:
	$$\frac{21}{23} = \uf{2} + \uf{4} + \uf{8} + \uf{32} + \uf{184} + \uf{736}.$$
\end{bsp}