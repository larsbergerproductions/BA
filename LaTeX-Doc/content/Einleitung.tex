% !TEX root = ../document.tex
\section{Einleitung}\label{sec:arithmetic}
Die aktuell ältesten mathematischen Aufzeichnungen der Menschheit sind zwei Papyrusrollen aus dem alten Ägypten, dem sogenannten Golenischev-Papyrus und dem Rhind-Papyrus, jeweils benannt nach ihrem letzten Besitzer, bevor sie in Museen gebracht wurden, die in etwa 4000 Jahre alt sind. Trotz dieses Alters wurden sie erst viel später entdeckt, aus ihren Bruchstücken wieder zusammengesetzt und übersetzt; das Rhind-Papyrus wurde in Teilen während Napoleon Bonapartes Invasion in Ägypten, die 1798 begann, gefunden, konnte aber erst 1922 vervollständigt und anschließend übersetzt werden. Voraussetzung für die Übersetzung dieser Schrift war aber das zuvor gewonnene Wissen über Hieroglyphen und die Demotische Schrift, einem Nachfolger der Hieratischen Schrift, die ihrerseits wieder eine kursive Variante der Hieroglyphen ist, vom sogenannten Stein von Rosette, einem Basalt, in den der gleiche Text in Hieroglyphen, Demotischer Schrift und Griechisch eingraviert ist. Dieser wurde 1799 von Offizieren der Armee Napoleons gefunden und diente danach dem Verständnis der bis dato unverständlichen Schriften des alten Ägypten.\cite[S.33 ff]{Burton2011}\\
Das Rhind Papyrus enthält in seiner Präambel folgende Worte, die den Inhalt des Dokuments erfassen sollen: ''Ein sorgfältiges Studium aller Dinge, Einblick in Alles, was es gibt, Wissen über alle obskuren Geheimnisse'' \footnote{\cite[S. 37, Übersetzung durch den Autor]{Burton2011}}. In den folgenden 85 Problemen werden dann Multiplikation und Division mittels wiederholter Addition beschrieben sowie Lösungsansätze für Probleme wie das Lösen von linearen Gleichungssystemen in einer Unbekannten.\\
Diese Arbeit soll sich mit einem sich aus der Division ergebenden Problemfeld befassen, nämlich dem der Ägyptischen Brüche. Zur thematischen Annäherung soll zunächst der Begriff ''Ägyptischer Bruch'' definiert werden.
\begin{def1}\label{def:egypfrac}
	Ein Bruch soll fortan ,,in ägyptischer Form'' \bzw\xspace,,Ägyptischer Bruch'' heißen genau dann, wenn er in der Form
	$$\uf{x_1} + \uf{x_2} + \cdots + \uf{x_n}, \quad n \in \N, n \ge 1$$
	mit paarweise verschiedenen $x_i$ vorliegt.
\end{def1}

Diese Art, Brüche zu notieren, wirft unzählige Fragen und Probleme auf, die zum Teil bis heute ungelöst sind. Es bietet sich dadurch ein breites Feld für theoretische und praktische Untersuchungen. Beispielsweise kann das mit Dezimalbrüchen relativ leicht zu erfassende Problem des Vergleichs zweier Brüche nicht auf Ägyptische Brüche übertragen werden, denn gibt es eine mögliche Darstellung für einen Ägyptischen Bruch, gibt es unendlich viele Varianten dafür, wodurch Operationen wie das Vergleichen zweier Ägyptischer Brüche oder das Berechnen von Quadratwurzeln solcher auf systematischem Wege schwierig bis unmöglich sind \cite[S. 62]{Resnikoff1984}.\\
In dieser Arbeit sollen verschiedene Fragen zum Thema ''Ägyptische Brüche'' behandelt und diskutiert werden. Zunächst soll der Ursprung und die Idee dieses Konzepts beleuchtet und begründet werden. Anschließend soll die Betrachtung einiger ausgewählter Algorithmen zur Umwandlung von Dezimalbrüchen in Ägyptische Brüche diese vergleichen und untersuchen, wie kompliziert eine solche Umwandlung sein kann, dazu wollen wir für jeden Algorithmus dessen Spezifikationen herausfiltern, die Algorithmen dann mit theoretischen Mitteln gegeneinander bezüglich ihrer Ergebnisse abschätzen und diese Abschätzungen anhand einer strukturellen Untersuchung mittels Computerrechnung überprüfen. Abschließend sollen bereits gefundene Zusammenhänge sowie noch nicht bewiesene Hypothesen aus dem Problemfeld der Ägyptischen Brüche beleuchtet und erklärt werden, um ein möglichst umfangreiches Bild von ebendiesem erfassen zu können.\\
\paragraph{Anmerkung}Die Divergenz der harmonischen Reihe zeigt, dass man mit Brüchen solcher Art durchaus alle rationalen Zahlen $x \in \Q$ erzeugen kann, auch ist es aber möglich, den ganzzahligen Anteil eines gemischten Bruchs gesondert zu betrachten, wodurch der Rest kleiner als 1 ist, weshalb hier auf eine Betrachtung von Brüchen $\frac{a}{b}\in \Q, \frac{a}{b} \geq 1$ verzichtet wird. Zudem werden Brüche $\frac{a}{b} \leq 0$ nicht beachtet, da dies historisch nicht relevant ist: die Ägypter kannten höchstwahrscheinlich keine negativen Zahlen, obwohl sie zumindest ein Symbol für ''Nichts'' besaßen, also wohl um die Existenz der Null wussten.