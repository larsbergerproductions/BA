% !TEX root = ../document.tex
\section{Einleitung}\label{sec:arithmetic}
Vor über 4000 Jahren entstand in Ägypten das heute als ''Rhind-Papyrus'' bekannte Dokument, das als älteste bekannte Schrift mathematischen Wissens der Menschheit gilt. In der Präambel dieses Papyrus heißt es ''Ein sorgfältiges Studium aller Dinge, Einblick in Alles, was es gibt, Wissen über alle obskuren Geheimnisse'' \footnote{\cite[S. 37, Übersetzung durch den Autor]{Burton2011}}.\\ In den dort enthaltenen 85 Problemen werden dann Multiplikation und Division definiert sowie darauf aufbauende Probleme diskutiert. Obwohl heute bekannt ist, dass die Arithmetik der Ägypter sich ab einem bestimmten Zeitpunkt nicht weiterentwickelte \bzw weiterentwickeln konnte, da sie \todo{insert ref} kein Stellenwertsystem besaßen, bietet die aus heutiger Sicht primitive Mathematik des alten Ägypten viele bis heute ungelöste Probleme. Ein solches Problemfeld sind die sogenannten Ägyptischen Brüche.

Das Beispiel der Ägyptischen Brüche hat einen sehr entscheidenden Nachteil: gibt es eine Möglichkeit der Darstellung einer rationalen Zahl als Summe von Stammbrüchen, gibt es unendlich viele solcher Möglichkeiten, somit ist eine Zerlegung nicht eindeutig bestimmt, was weiteres Verfahren mit solchen Brüchen, beispielsweise schon das Vergleichen zweier Brüche oder das Bestimmen der Quadratwurzeln solcher Brüche auf einem systematischen Weg schwer bis unmöglich macht. \cite[S.62]{Resnikoff1984}
\\...

\begin{def1}\label{def:egypfrac}
	Ein Bruch soll fortan ,,in ägyptischer Form'' \bzw  ,,Ägyptischer Bruch'' heißen genau dann, wenn er in der Form
	$$\uf{x_1} + \uf{x_2} + \cdots + \uf{x_n}, \quad n \in \N, n \ge 1$$
	mit paarweise verschiedenen $x_i$ vorliegt.
\end{def1}

Obwohl die Divergenz der harmonischen Reihe zeigt, dass man mit Brüchen solcher Art durchaus alle rationalen Zahlen $x \in \Q$ erzeugen kann, waren für ganzzahlige Werte in Ägypten Schreibweisen gängig, weshalb hier auf eine Betrachtung von Brüchen $\frac{a}{b}\in \Q, \frac{a}{b} \geq 1$ verzichtet wird. Zudem werden Brüche $\frac{a}{b} \leq 0$ nicht beachtet, da dies historisch nicht relevant ist: die Ägypter kannten höchstwahrscheinlich keine negativen Zahlen, obwohl sie zumindest ein Symbol für ''Nichts'' besaßen, also wohl um die Existenz der Null wussten.