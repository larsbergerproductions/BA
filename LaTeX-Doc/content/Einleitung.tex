% !TEX root = ../document.tex
\section{Einleitung}
Vor über 4000 Jahren entstand in Ägypten das heute als ''Rhind-Papyrus'' bekannte Dokument, das als älteste bekannte Schrift mathematischen Wissens der Menschheit gilt. In der Präambel dieses Papyrus heißt es ''Ein sorgfältiges Studium aller Dinge, Einblick in Alles, was es gibt, Wissen über alle obskuren Geheimnisse'' \cite[S. 37, Übersetzung durch den Autor]{Burton2011}\\ In den dort enthaltenen 85 Problemen werden dann Multiplikation und Division definiert sowie darauf aufbauende Probleme diskutiert. Obwohl heute bekannt ist, dass die Arithmetik der Ägypter sich ab einem bestimmten Zeitpunkt nicht weiterentwickelte \bzw weiterentwickeln konnte, da sie \todo{insert ref} kein Stellenwertsystem besaßen, bietet die aus heutiger Sicht primitive Mathematik des alten Ägypten viele bis heute ungelöste Probleme. Ein solches Problemfeld sind die sog. Ägyptischen Brüche.
\\...

\begin{def1}
	Ein Bruch soll fortan ,,in ägyptischer Form'' \bzw  ,,Ägyptischer Bruch'' heißen, genau dann wenn er in der Form
	$$\uf{x_1} + \uf{x_2} + \cdots + \uf{x_n}, \quad n \in \N$$
	vorliegt.
\end{def1}

Obwohl die Divergenz der Harmonischen Reihe zeigt, dass man mit Brüchen solcher Art durchaus alle Rationalen Zahlen $x \in \Q$ erzeugen kann, waren für ganzzahlige Werte in Ägypten Schreibweisen gängig, weshalb hier auf eine Betrachtung von Brüchen $\frac{a}{b} \geq 1$ verzichtet wird.