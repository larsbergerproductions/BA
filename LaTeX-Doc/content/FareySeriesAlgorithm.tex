\subsection{Der Farey-Folgen-Algorithmus}
Eine weitere Methode zur Erstellung Ägyptischer Brüche stellt der sogenannte Farey-Folgen-Algorithmus dar, der seinen Namen aus dem Umstand bezieht, dass die Farey-Folge dafür genutzt wird.
\begin{def1}
	Sei $q \in \N$. Die Farey-Folge der Ordnung $q$, $\, F_q$, ist definiert als die aufsteigend sortierte Folge aller einmalig darin vorkommenden gekürzten Brüche $\frac{a}{b} \in \Q$, für die gilt:
	$0\leq a \leq b \leq q,\, b\neq 0$.
\end{def1}

\begin{bsp}
	Sei $q=5$. Die Farey-Folge der Ordnung 5 ist
	$$F_5 = \left\{\frac{0}{1}, \uf{5}, \uf{4}, \uf{3}, \frac{2}{5}, \uf{2}, \frac{3}{5}, \frac{2}{3}, \frac{3}{4}, \frac{4}{5}, \uf{1} \right\}.$$
\end{bsp}

Der Algorithmus funktioniert dann wie folgt.
\begin{algorithm}\label{algo:FareySeries}
	Sei $\frac{p}{q} \in \Q_+$ in reduzierter Form der zu zerlegende Bruch.
	\begin{enumerate}
		\item Konstruiere $F_q$.
		\item Sei $\frac{r}{s}$ der zu $\frac{p}{q}$ adjazente Bruch in $F_q$, sodass $\frac{r}{s} < \frac{p}{q}$. Aufgrund der Eigenschaften der Farey-Folge gilt dann
		$$\frac{p}{q} = \uf{qs} + \frac{r}{s},$$ wobei $s<q,\, r<p$ \cite[S. 425]{Beck2000}.
		\item Wiederhole dieses Vorgehen für $\frac{r}{s}$ solange, bis $s=1 \gdw r=0$.
	\end{enumerate}
\end{algorithm}
\begin{satz}
	Algorithmus \ref{algo:FareySeries} terminiert für jede Eingabe.
\end{satz}
\begin{bew}
	Sei $\frac{p}{q} \in \Q_+$ der zu zerlegende rationale Bruch, $\frac{r}{s}, \frac{t}{u} \in \Q_+$ die zu $\frac{p}{q}$ in $F_q$ adjazenten Brüche, wobei gilt $\frac{r}{s}<\frac{p}{q} < \frac{t}{u}$.\\
	Es gilt $$\frac{p}{q} = \uf{qs}+\frac{r}{s}.$$
	Falls $r = 0$, dann $\frac{p}{q} = \uf{qs}$ und der Algorithmus terminiert.
	Sonst wird der Algorithmus für $\frac{r}{s}$ in $F_s$ wiederholt. Es gilt $s<q$, da kein Nenner in $F_q$ größer als $q$ ist und zwei beliebige Brüche mit demselben Nenner niemals adjazent zueinander sind. Somit wird nach endlich vielen Schritten $r=0$ erreicht. Da in jeder Farey-Folge der einzige Bruch mit Zähler Null $\frac{0}{1}$ ist, ist dies der Abbruchfall.
\end{bew}

Da die Farey-Folge $F_q$ schon für mäßig große $q$ sehr groß wird, bietet sich für das tatsächliche Berechnen eine Optimierung an, indem nur der relevante Teil der Farey-Folge konstruiert wird. Anwendung findet dabei das Bisektionsverfahren.
\begin{def1}\label{def:mediant}
	Die im Folgenden verwendete Mediante zweier rationaler Brüche $\frac{a}{b}, \frac{c}{d} \in \Q_+$ $mediant: \Q_+ \times \Q_+ \rightarrow \Q_+$ sei definiert als:
	$$mediant \left(\frac{a}{b}, \frac{c}{d}\right) = \frac{a+c}{b+d}.$$
\end{def1}
\begin{bsp}\label{bsp:Frel}
	Sei $\frac{p}{q} = \frac{21}{23}$. Die obere und untere Schranken sind in unserem Fall $0$ \bzw $1$. Die Mediante liegt also bei $\uf{2}$, wir stellen fest: $\uf{2}<\frac{21}{23}<1$, also setzen wir die Suche im Intervall $[\uf{2}, 1]$ fort. Die Mediante liegt nun bei $\frac{2}{3}$, $\frac{2}{3}<\frac{21}{23}<1$ \usw
	Bei der Mediante $\frac{11}{12}$ stellen wir fest: $\frac{10}{11} < \frac{21}{23} < \frac{11}{12}.$ Daraus folgt der relevante Teil von $F_{23}$, hier $F_{23rel}$ genannt: $$F_{23rel} = \left\{\frac{0}{1}, \frac{1}{2}, \frac{2}{3}, \frac{3}{4}, \frac{4}{5}, \frac{5}{6}, \frac{6}{7}, \frac{7}{8}, \frac{8}{9}, \frac{9}{10}, \frac{10}{11}, \frac{21}{23}, \frac{11}{12}\right\}.$$
	Weil für die Adjazenzberechnung von $\frac{p}{q}$ nur die unmittelbare Umgebung nötig ist, fallen alle Brüche, die größer als der größere der zu $\frac{p}{q}$ adjazenten Brüche sind, aus der gekürzten Farey-Folge heraus; im Beispiel betrifft dies $\uf{1}$.
	Somit enthält die gekürzte Farey-Folge nur noch 12 der andererseits 173 zu berechnenden Elemente von $F_{23}$.
\end{bsp}

\todo[inline]{Beispielrechnung}