\section[Theoretische Schranken]{Theoretische Schranken für die maximale Anzahl der Stammbrüche}\label{sec:theorie}
Nachdem untersucht wurde, was in der Praxis mit ausgewählten Methoden erreicht werden kann, soll nun das Licht auf einen weiteren sehr interessanten Aspekt Ägyptischer Brüche gelenkt werden, nämlich den Dingen, die theoretische herleitbare Schranken haben. Vor allem im 20. Jahrhundert wurde von zahlreichen Mathematikern an dem Problem, solche theoretischen Grenzen zu finden und zu beweisen, gearbeitet, unter ihnen Bleicher, Erdös, Graham und andere \cite[S.87 ff]{Guy1981}.

\begin{satz}\label{satz:two/n}
	Sei $n \in \N$ ungerade. $\frac{2}{n}$ lässt sich für jedes $n$ als Summe zweier Stammbrüche notieren, nämlich:
	$$\frac{2}{n} = \uf{\lceil \frac{n}{2} \rceil} + \uf{n \cdot \lceil \frac{n}{2} \rceil}.$$
\end{satz}
\begin{bew}
	Sei $m \in \N$ so gewählt, dass $n=2m+1$. Es folgt:
	\begin{eqnarray*}
		\uf{\lceil \frac{n}{2} \rceil} + \uf{n \cdot \lceil \frac{n}{2} \rceil} & = & \uf{m+1} + \uf{(2m+1)(m+1)}\\
		& = & \frac{2m+1}{(2m+1)(m+1)} + \frac{1}{(2m+1)(m+1)}\\
		& = & \frac{2m+2}{(2m+1)(m+1)}\\
		& = & \frac{2(m+1)}{(2m+1)(m+1)}\\
		& = & \frac{2}{2m+1}\\
		& = & \frac{2}{n}.
	\end{eqnarray*}
\end{bew}

Da der Term $\uf{\lceil \frac{n}{2} \rceil}$ genau dem größten Stammbruch entspricht, der kleiner als $\frac{2}{n}$ ist, liefert der Greedy-Algorithmus hier für alle $\frac{2}{n}$ das gleiche Ergebnis wie die Rechenvorschrift aus Satz \ref{satz:two/n}. Damit lässt sich auch sofort eine weiterer Satz aufstellen, diesmal für $\frac{3}{n}$.

\begin{satz}
	Für jedes beliebige $n \in \N$ lässt sich $\frac{3}{n}$ als Summe von höchstens 3 Stammbrüchen schreiben.
\end{satz}
\begin{bew}
	Es gilt
	$$\frac{3}{n} = \uf{n} + \frac{2}{n}.$$
	Falls $n$ gerade ist, reichen sogar nur zwei Stammbrüche, denn dann kann $\frac{2}{n} = \uf{\frac{n}{2}}$ geschrieben werden; anderenfalls wird $\frac{2}{n}$ wie nach Satz \ref{satz:two/n} zerlegt und es ergeben sich genau drei Stammbrüche:
	$$\frac{3}{n} = \uf{n} + \uf{\lceil \frac{n}{2} \rceil} + \uf{n \cdot \lceil \frac{n}{2} \rceil}.$$
\end{bew}

Bleicher und Erdös vermuteten, dass für alle $n \in \N$ der Bruch $\frac{4}{n}$ in höchstens 3 Stammbrüche zerlegt werden kann, was nicht formal bewiesen, aber von Nicola Franceschine zumindest für $n < 10^8$ gezeigt wurde. Die gleiche Vermutung für $\frac{5}{n}$ wurde durch W. Sierpiński aufgestellt, zunächst durch G. Palamá für alle $n < 922.321$ gezeigt und später durch Stewart auf alle $n < 1.057.438.801$, $n \not\equiv 1\, (\text{mod } 278.460)$ erweitert.
Schinzel bewies zudem, dass der Ausdruck
$$\frac{4}{at+b} = \uf{x(t)} + \uf{y(t)} + \uf{z(t)}$$
für Polynome $x(t), y(t)$ und $z(t)$ mit ganzzahligen Koeffizienten genau dann gilt, wenn $b$ nicht quadratischer Rest modulo $a$ ist. \cite[S. 88]{Guy1981}
\todo[inline]{mehr Theorie!!}
Bereits das vergleichsweise einfache Beispiel $\frac{4}{n}$, für welches noch immer nicht klar ist, ob es sich für alle $n \in \N$ durch höchstens 3 Stammbrüche als Ägyptischer Bruch darstellen lässt, zeigt auf, wie viel in diesem Bereich der Mathematik noch unbekannt ist. Das Problemfeld der Ägyptischen Brüche, das vor knapp 4000 Jahren aufgeworfen wurde, stellt auch die heutigen Mathematiker noch vor viele ungelöste Aufgaben und Fragen.