\section{Zusammenfassende Auswertung}
In den vorangegangenen Kapiteln wurde das Thema ,,Ägyptische Brüche'' eingeführt und die ursprüngliche Motivation für diese Zahldarstellung aufgezeigt. Nach einem Einblick in eine Auswahl von Zerlegungsalgorithmen für die Erstellung solcher Brüche wurden diese verglichen und es konnten einige interessante und aufschlussreiche Erkenntnisse bezüglich der Arbeitsweise und Effizienz dieser Algorithmen gewonnen werden. Natürlich gibt es weitaus mehr als die aufgezeigten Algorithmen, auch ist es nicht unwahrscheinlich, dass neue hinzukommen werden, die unter Umständen noch effizienter arbeiten bzw. auch bessere theoretische Schranken haben. Außerdem wurde ein Ausblick hinter die theoretischen Kulissen gegeben, der allerdings längst nicht ausreicht, um alle möglichen Forschungsfelder dieses Bereichs abzubilden. Bis heute forschen Mathematiker weiter an den Ägyptischen Brüchen, auch wenn diese in der Praxis schon längst keine Anwendung mehr haben. Nichtsdestotrotz kann man weitere interessante und spannende Ergebnisse und neue Beobachtungen in diesem Bereich der elementaren Zahlentheorie erwarten. Insofern gab diese Arbeit einen Überblick und Ausblick bezüglich des vielfältigen und weitreichenden Themenfeldes der Ägyptischen Brüche.