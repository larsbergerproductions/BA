% Nach Vorlage der Projektdokumentation für Fachinformatiker Anwendungsentwicklung von Stefan Macke (http://fiae.link/LaTeXVorlageFIAE)
% Deckblatt der UniBwM Fakultät für Informatik 
% Mathebefehle von Dr. Sören Kleine UniBwM
\documentclass[
	ngerman,
	11pt,
	toc=listof,
	toc=bibliography,
	footnotes=multiple,
	parskip=half,
	numbers=noendperiod
	%,twocolumn
]{scrartcl}

\usepackage[utf8]{inputenc}

% !TEX root = document.tex

% Hinweis: der Titel muss zum Inhalt des Projekts passen und den zentralen Inhalt des Projekts deutlich herausstellen
\newcommand{\titel}{Darstellung rationaler Zahlen durch\\ Ägyptische Brüche}
\newcommand{\untertitel}{Eine Untersuchung von Algorithmen und Aufwand}
\newcommand{\kurztitel}{Ägyptische Brüche}
\newcommand{\kompletterTitel}{\titel{} -- \untertitel}

\newcommand{\autorName}{Lars Berger}
\newcommand{\autorNameKurz}{Berger}
\newcommand{\autorAnschrift}{Münchner Straße 7}
\newcommand{\autorOrt}{82008 Unterhaching}
\newcommand{\matrikelnummer}{1173278}

\newcommand{\betriebLogo}{images/unibw.jpg}
\newcommand{\betriebName}{UniBw}		%z.B. UniBw
\newcommand{\betriebAnschrift}{Werner-Heisenberg-Weg 39}
\newcommand{\betriebOrt}{85579 Neubiberg}

\newcommand{\betreff}{Bachelorarbeit}
\newcommand{\pruefungstermin}{}
\newcommand{\abgabeOrt}{Neubiberg}
\newcommand{\abgabeTermin}{19.11.2019}

\newcommand{\arbeitsTyp}{Bachelorarbeit}	% z.B. Bacheloararbeit, Mitschrift, Hausarbeit

\newcommand{\aufgabensteller}{Prof. Dr. Cornelius Greither}
\newcommand{\betreuer}{Dr. Soeren Kleine}

% !TEX root = ../document.tex

% Anpassung an Landessprache ---------------------------------------------------
\usepackage{babel}

% Umlaute ----------------------------------------------------------------------
%   Umlaute/Sonderzeichen wie äüöß direkt im Quelltext verwenden (CodePage).
%   Erlaubt automatische Trennung von Worten mit Umlauten.
% ------------------------------------------------------------------------------
\usepackage[T1]{fontenc}
\usepackage{textcomp} % Euro-Zeichen etc.

% Schrift ----------------------------------------------------------------------
\usepackage{lmodern} % bessere Fonts
\usepackage{relsize} % Schriftgröße relativ festlegen
\usepackage{dsfont}  % math symbols etc.
\usepackage[math]{cellspace}
\usepackage{pifont}
\usepackage[pdftex, dvipsnames]{xcolor}
% Tabellen ---------------------------------------------------------------------
\PassOptionsToPackage{table}{xcolor}
\usepackage{tabularx}
% für lange Tabellen
\usepackage{longtable}
\usepackage{array}
\usepackage{ragged2e}
\usepackage{lscape}
\newcolumntype{w}[1]{>{\raggedleft\hspace{0pt}}p{#1}} % Spaltendefinition rechtsbündig mit definierter Breite

% Grafiken ---------------------------------------------------------------------
\usepackage[dvips,final]{graphicx} % Einbinden von JPG-Grafiken ermöglichen
\usepackage{graphics} % keepaspectratio
\usepackage{wrapfig}
\usepackage{floatflt} % zum Umfließen von Bildern
\usepackage{float}
\usepackage{subcaption}
\graphicspath{{Bilder/}} % hier liegen die Bilder des Dokuments

% Fußnoten ---------------------------------------------------------------------
\let\oldfootnote\footnote
\renewcommand\footnote[1]{\oldfootnote{\hspace{2mm}#1}}

% Sonstiges --------------------------------------------------------------------
\usepackage[titles]{tocloft} % Inhaltsverzeichnis DIN 5008 gerecht einrücken
\usepackage{amsmath,amsfonts,amssymb,amsthm} % Befehle aus AMSTeX für mathematische Symbole
\usepackage{lipsum}
\usepackage{enumitem} % anpassbare Enumerates/Itemizes
\usepackage{xspace} % sorgt dafür, dass Leerzeichen hinter parameterlosen Makros nicht als Makroendezeichen interpretiert werden

\usepackage{makeidx} % für Index-Ausgabe mit \printindex
\usepackage[printonlyused]{acronym} % es werden nur benutzte Definitionen aufgelistet

% Einfache Definition der Zeilenabstände und Seitenränder etc.
\usepackage{setspace}
\usepackage[margin = 2in]{geometry}

%lokale Definition von Mehrspalten-layouts
%\usepackage{minipage-marginpar}

% Symbolverzeichnis
\usepackage[intoc]{nomencl}
\let\abbrev\nomenclature
\renewcommand{\nomname}{Abkürzungsverzeichnis}
\setlength{\nomlabelwidth}{.25\hsize}
\renewcommand{\nomlabel}[1]{#1 \dotfill}
\setlength{\nomitemsep}{-\parsep}

\usepackage{varioref} % Elegantere Verweise. „auf der nächsten Seite“
\usepackage{url} % URL verlinken, lange URLs umbrechen etc.

\usepackage{chngcntr} % fortlaufendes Durchnummerieren der Fußnoten
% \usepackage[perpage]{footmisc} % Alternative: Nummerierung der Fußnoten auf jeder Seite neu

\usepackage{ifthen} % bei der Definition eigener Befehle benötigt
\usepackage[colorinlistoftodos,prependcaption,textsize=tiny]{todonotes} % definiert u.a. die Befehle \todo und \listoftodos
\usepackage[square]{natbib} % wichtig für korrekte Zitierweise
\usepackage{xargs}
% PDF-Optionen -----------------------------------------------------------------
\usepackage{pdfpages}
\pdfminorversion=5 % erlaubt das Einfügen von pdf-Dateien bis Version 1.7, ohne eine Fehlermeldung zu werfen (keine Garantie für fehlerfreies Einbetten!)
\usepackage[
    bookmarks,
    bookmarksnumbered,
    bookmarksopen=true,
    bookmarksopenlevel=1,
    colorlinks=true,
% diese Farbdefinitionen zeichnen Links im PDF farblich aus
%    linkcolor=AOBlau, % einfache interne Verknüpfungen
%    anchorcolor=AOBlau,% Ankertext
%    citecolor=AOBlau, % Verweise auf Literaturverzeichniseinträge im Text
%    filecolor=AOBlau, % Verknüpfungen, die lokale Dateien öffnen
%    menucolor=AOBlau, % Acrobat-Menüpunkte
%    urlcolor=AOBlau,
% diese Farbdefinitionen sollten für den Druck verwendet werden (alles schwarz)
    linkcolor=black, % einfache interne Verknüpfungen
    anchorcolor=black, % Ankertext
    citecolor=black, % Verweise auf Literaturverzeichniseinträge im Text
    filecolor=black, % Verknüpfungen, die lokale Dateien öffnen
    menucolor=black, % Acrobat-Menüpunkte
    urlcolor=black,
%
    %backref, % Quellen werden zurück auf ihre Zitate verlinkt
    pdftex,
    plainpages=false, % zur korrekten Erstellung der Bookmarks
    pdfpagelabels=true, % zur korrekten Erstellung der Bookmarks
    hypertexnames=false, % zur korrekten Erstellung der Bookmarks
    linktocpage % Seitenzahlen anstatt Text im Inhaltsverzeichnis verlinken
]{hyperref}
% Befehle, die Umlaute ausgeben, führen zu Fehlern, wenn sie hyperref als Optionen übergeben werden
\hypersetup{
    pdftitle={\titel -- \untertitel},
    pdfauthor={\autorName},
    pdfcreator={\autorName},
    pdfsubject={\titel -- \untertitel},
    pdfkeywords={\titel -- \untertitel},
}


% zum Einbinden von Programmcode -----------------------------------------------
\usepackage{listings}

\definecolor{hellgelb}{rgb}{1,1,0.9}
\definecolor{colKeys}{rgb}{0,0,1}
\definecolor{colIdentifier}{rgb}{0,0,0}
\definecolor{colComments}{rgb}{0,0.5,0}
\definecolor{colString}{rgb}{1,0.5,0}
\lstset{
    float=hbp,
	basicstyle=\footnotesize,
    identifierstyle=\color{colIdentifier},
    keywordstyle=\color{colKeys},
    stringstyle=\color{colString},
    commentstyle=\color{colComments},
    backgroundcolor=\color{hellgelb},
    columns=flexible,
    tabsize=2,
    frame=single,
    extendedchars=true,
    showspaces=false,
    showstringspaces=false,
    numbers=left,
    numberstyle=\tiny,
    breaklines=true,
    breakautoindent=true,
	captionpos=b,
}
\lstdefinelanguage{cs}{
	sensitive=false,
	morecomment=[l]{//},
	morecomment=[s]{/*}{*/},
	morestring=[b]",
	morekeywords={
		abstract,event,new,struct,as,explicit,null,switch
		base,extern,object,this,bool,false,operator,throw,
		break,finally,out,true,byte,fixed,override,try,
		case,float,params,typeof,catch,for,private,uint,
		char,foreach,protected,ulong,checked,goto,public,unchecked,
		class,if,readonly,unsafe,const,implicit,ref,ushort,
		continue,in,return,using,decimal,int,sbyte,virtual,
		default,interface,sealed,volatile,delegate,internal,short,void,
		do,is,sizeof,while,double,lock,stackalloc,
		else,long,static,enum,namespace,string},
}
\lstdefinelanguage{natural}{
	sensitive=false,
	morecomment=[l]{/*},
	morestring=[b]",
	morestring=[b]',
	alsodigit={-,*},
	morekeywords={
		DEFINE,DATA,LOCAL,END-DEFINE,WRITE,CALLNAT,PARAMETER,USING,
		IF,NOT,END-IF,ON,*ERROR-NR,ERROR,END-ERROR,ESCAPE,ROUTINE,
		PERFORM,SUBROUTINE,END-SUBROUTINE,CONST,END-FOR,END,FOR,RESIZE,
		ARRAY,TO,BY,VALUE,RESET,COMPRESS,INTO,EQ},
}
\lstdefinelanguage{php}{
	sensitive=false,
	morecomment=[l]{/*},
	morestring=[b]",
	morestring=[b]',
	alsodigit={-,*},
	morekeywords={
		abstract,and,array,as,break,case,catch,cfunction,class,clone,const,
		continue,declare,default,do,else,elseif,enddeclare,endfor,endforeach,
		endif,endswitch,endwhile,extends,final,for,foreach,function,global,
		goto,if,imlements,interface,instanceof,namespace,new,old_function,or,
		private,protected,public,static,switch,throw,try,use,var,while,xor
		die,echo,empty,exit,eval,include,include_once,isset,list,require,
	false	require_once,return,print,unset},
}
\lstdefinelanguage{gp}{
	sensitive=true,
	morecomment=[s]{/*}{*/},
	morestring=[b]',
	morestring=[b]",
	morekeywords={
		local, listput, print, return, while, if, sum
	},
}

% Glossar -------------------------------------------------------------------------
\usepackage[toc,section,style=altlist,nonumberlist]{glossaries}
\makeglossaries

\input{TeX-StyleDefinition/sitestyle}
\input{TeX-StyleDefinition/commands}
% !TEX root = ../document.tex

\newcommand{\R}{\mathds{R}}
\newcommand{\Z}{\mathds{Z}}
\newcommand{\N}{\mathds{N}}
\newcommand{\Q}{\mathds{Q}}
\newcommand{\K}{\mathds{K}}
\newcommand{\C}{\mathds{C}}
\newcommand{\B}{\mathds{B}}
\newcommand{\F}{\mathds{F}}
\newcommand{\p}{\mathfrak{p}}
\newcommand{\Pot}{\mathcal{P}}
\newcommand{\id}{\textup{id}}
\newcommand{\Ker}{\textup{Ker}}
\newcommand{\Image}{\textup{Im}}
\newcommand{\la}{\langle}
\newcommand{\ra}{\rangle}
\newcommand{\gdw}{\Leftrightarrow}

\newtheorem{lemma}{Lemma}[subsection]
\newtheorem{prop}[lemma]{Proposition}
\newtheorem{defprop}[lemma]{Definition and Proposition}
\newtheorem{satz}[lemma]{Satz}
\newtheorem{thm}[lemma]{Theorem} 
\newtheorem{kor}[lemma]{Korollar} 
\newtheorem{folg}[lemma]{Folgerung}
\newenvironment{bew}{\begin{proof}[Beweis]}{\end{proof}}

\theoremstyle{definition} 
\newtheorem{def1}[lemma]{Definition} 
\newtheorem{bem}[lemma]{Bemerkung}
\newtheorem{bsp}[lemma]{Beispiel}
\newtheorem{notation}[lemma]{Notation}
\newtheorem{algorithm}[lemma]{Algorithmus}

\newcommand{\uf}[1]{\frac{1}{#1}} % unit fraction \frac{1}{n}

%for the \usepackage[math]{cellspace}
\cellspacetoplimit 2pt
\cellspacebottomlimit 2pt
\input{Commands}
%\newglossaryentry{Referenz}{name=Referenz,description={Eine Referenz ist \ldots}}


\begin{document}
	
% ######################################
% #                                    #
% #              DECKBLATT             #
% #                                    #
% ######################################	
	\phantomsection						
	\thispagestyle{plain}
	\pagenumbering{gobble}	
	\pdfbookmark[1]{Deckblatt}{deckblatt}
	% !TEX root = document.tex
\begin{titlepage}
	\centering
	
	\includegraphics[width=0.5\textwidth]{\betriebLogo}
	\\
	\vspace{2cm}
	\LARGE{\textsf{\arbeitsTyp}}
	\\
	\vspace{2cm}
	\LARGE{\textsf{\textbf{\titel}}}
	\\
	\vspace{1cm}
	\Large{\textsf{\textbf{\untertitel}}}
	\\
	\vfill
	\textsf{
		\begin{normalsize}
			\begin{tabular}{ll}
				Eingereicht von: 	& \autorName\\
									& \matrikelnummer\\
				\\
				Aufgabensteller: 	& \aufgabensteller \\                           
				\\
				Betreuer: 			& \betreuer \\                       
				\\	        
				\\		 
				Abgabedatum: 		& \abgabeTermin\\
			\end{tabular}        
		\end{normalsize}    
	}
	\\
	\vfill
	\normalsize{\textsf{Universität der Bundeswehr München\\Fakultät für Informatik\\Institut für Mathematik und Operations Research}}
	
\end{titlepage}


	\cleardoublepage
	
	
% ######################################
% #                                    #
% #         INHALTSVERZEICHNIS         #
% #                                    #
% ######################################	
	\phantomsection						
	\pdfbookmark[1]{Inhaltsverzeichnis}{inhalt}
	\pagenumbering{Roman}
	\tableofcontents
	\cleardoublepage
% ######################################
% #                                    #
% #               INHALT               #
% #                                    #
% ######################################
	\pagenumbering{arabic}
	% !TEX root = document.tex
%\input{Inhalt/(...)}
\listoftodos[TODOs]
\clearpage
% !TEX root = ../document.tex
\section{Einleitung}
Vor über 4000 Jahren entstand in Ägypten das heute als ''Rhind-Papyrus'' bekannte Dokument, das als älteste bekannte Schrift mathematischen Wissens der Menschheit gilt. In der Präambel dieses Papyrus heißt es ''Ein sorgfältiges Studium aller Dinge, Einblick in Alles, was es gibt, Wissen über alle obskuren Geheimnisse'' \cite[S. 37, Übersetzung durch den Autor]{Burton2011}\\ In den dort enthaltenen 85 Problemen werden dann Multiplikation und Division definiert sowie darauf aufbauende Probleme diskutiert. Obwohl heute bekannt ist, dass die Arithmetik der Ägypter sich ab einem bestimmten Zeitpunkt nicht weiterentwickelte \bzw weiterentwickeln konnte, da sie \todo{insert ref} kein Stellenwertsystem besaßen, bietet die aus heutiger Sicht primitive Mathematik des alten Ägypten viele bis heute ungelöste Probleme. Ein solches Problemfeld sind die sog. Ägyptischen Brüche.
\\...

\begin{def1}
	Ein Bruch soll fortan ,,in ägyptischer Form'' \bzw  ,,Ägyptischer Bruch'' heißen, genau dann wenn er in der Form
	$$\uf{x_1} + \uf{x_2} + \cdots + \uf{x_n}, \quad n \in \N$$
	vorliegt.
\end{def1}

Obwohl die Divergenz der Harmonischen Reihe zeigt, dass man mit Brüchen solcher Art durchaus alle Rationalen Zahlen $x \in \Q$ erzeugen kann, waren für ganzzahlige Werte in Ägypten Schreibweisen gängig, weshalb hier auf eine Betrachtung von Brüchen $\frac{a}{b} \geq 1$ verzichtet wird.
\clearpage
% !TEX root=../document.tex
\section{Ägyptische Arithmetik}
	Um die Verwendung ägyptischer Brüche historisch zu verstehen, lohnt sich ein kurzer Blick in die Arithmetik des alten Ägypten. Im Folgenden werden dazu die Multiplikation und die darauf aufbauende Division betrachtet, welche letztendlich die Verwendung von Brüchen bei Division mit nichttrivialem Rest erforderlich macht. Das gesamte arithmetische System der Ägypter baut dabei letztendlich auf der Addition auf.

\subsection{Ägyptische Multiplikation}\label{subsec:egypMult}
	Bei der Multiplikation zweier ganzer Zahlen $a \cdot b$ mit $a, b \in \N$ stellt man eine zweispaltige Tabelle auf, die in der linken Spalte die Zweierpotenzen $2^{n}$ für die n-te Zeile, mit n=0 beginnend, und rechts das Produkt $a \cdot 2^n$.
	Zur Multiplikation wählt man nun aus der linken Spalte die Zeilen aus, deren Werte sich zu $b$ addieren, und markiert diese, bspw. mittels eines Hakens (\checkmark). Schließlich werden die Zahlen der rechten Spalte aufaddiert, deren Zeile mittels \checkmark markiert wurde. Die sich ergebende Summe ist das Ergebnis.
	
	\begin{bsp}\label{Bsp: EgypMult}
		Die Multiplikation $\textcolor{OliveGreen}{23} \cdot \textcolor{blue}{69}$ bzw. $\textcolor{blue}{69} \cdot \textcolor{OliveGreen}{23}$ exemplarisch:\\
		\begin{minipage}{.5\textwidth}
				\begin{center}
				\begin{tabular}{r r r}
					&&\\
					&&\\
					\checkmark &1 & \textcolor{blue}{69}\\
					\checkmark &2 & 138\\
					\checkmark &4 & 276\\
					&8 & 552\\
					\checkmark &16 & 1104\\ \hline
					Summe: &\textcolor{OliveGreen}{23} & \underline{\underline{1587}}\\
				\end{tabular}
			\end{center}
		\end{minipage}
		\begin{minipage}{.5\textwidth}
			\begin{center}
				\begin{tabular}{r r r}
					\checkmark & 1 & \textcolor{OliveGreen}{23}\\
					& 2 & 46\\
					\checkmark & 4 & 92\\
					& 8 & 184\\
					& 16 & 368\\
					& 32 & 736\\
					\checkmark & 64 & 1472\\ \hline
					Summe: & \textcolor{blue}{69} & \underline{\underline{1587}}\\
				\end{tabular}
			\end{center}
		\end{minipage}
	
	\end{bsp}
	
	Diese Methode funktioniert, da jede natürliche Zahl $n \in \N$ als Summe paarweise verschiedener Zweierpotenzen darstellbar ist. Es wird im Allgemeinen angezweifelt, dass dies von den Ägyptern je formal bewiesen wurde, aber die Nutzung zeigt, dass sie diesen Zusammenhang zumindest erkannt hatten. \cite{Burton2011}
\subsection{Ägyptische Division}
	\subsubsection{Ganzzahlige Division}
	Der einfachere Fall der ganzzahligen Division ohne Rest ist dem Prinzip der Multiplikation sehr ähnlich, nur die Herangehensweise ist verändert. Zur Division von $a \div b$ mit $a, b \in \N$ nutzt man die Tabellenmethode aus der Multiplikation wie in \ref{subsec:egypMult}, nur wird $a$ als Ergebnis in der letzten Zeile rechts notiert und der fehlende Multiplikator ergibt sich dann in der letzten Zeile links.
	
	\begin{bsp}
		Es sei die Division $\textcolor{OliveGreen}{117} \div \textcolor{blue}{9}$ betrachtet. Die Tabelle generiert sich wie oben. Nun wählt man alle Zeilen aus, deren rechte Spalten sich zu $117$ addieren, addiert die linke Spalte der ausgewählten Zeilen und erhält das Ergebnis $117 \div 9 = 13$
		\begin{center}
			\begin{tabular}{r r r}
				\checkmark & 1 & \textcolor{blue}{9}\\
				& 2 & 18\\
				\checkmark & 4 & 36\\
				\checkmark & 8 & 72\\ \hline
				& \underline{\underline{13}} & \textcolor{OliveGreen}{117}
			\end{tabular}
		\end{center}
	\end{bsp}

	\subsubsection{Division mit Rest}
	Die Division mit Rest $a \div b$ mit $a, b \in \N$ funktioniert ähnlich wie die ganzzahlige, jedoch fügt man an die Tabelle nun noch die nötigen Bruchteile von $a$ an, die nötig sind.
	
	\begin{bsp}
		Wir betrachten hierfür die Division $\textcolor{blue}{117} \div \textcolor{OliveGreen}{7}$.
		\begin{center}
			\begin{tabular}{r r r}
				& 1 & \textcolor{OliveGreen}{7}\\
				& 2 & 14\\
				& 4 & 28\\
				& 8 & 56\\
				\checkmark & 16 & 112\\
				\checkmark & $\uf{2}$ & $3+\uf{2}$\\
				\checkmark & $\uf{7}$ & $1$\\
				\checkmark & $\uf{14}$ & $\uf{2}$\\ \hline
				& \underline{\underline{$16+\uf{2}+\uf{7}+\uf{14}$}} & \textcolor{blue}{$117$}
			\end{tabular}
		\end{center}
	\end{bsp}
	Sei $x \in \N$ und $n$ der Divisor der gewählten Division. Typische Brüche, die in der linken Spalte verwendet wurden, weil einfach zu berechnen, waren $\uf{2x}$ sowie $\uf{nx}$.
	
	Aus dieser Methodik heraus ergibt sich die Notation der Ägyptischen Brüche. Selbstverständlich wurde auch mit solchen Brüchen multipliziert und dividiert, solche Beispiele würden aber den Rahmen dieser Arbeit sprengen, weshalb diese nicht weiter betrachtet werden.
	
\subsection{Ermittlung Ägyptischer Zerlegungen von Brüchen}
Die Ägypter brauchten nun also ein System, mit dessen Hilfe sie die Zerlegung von Brüchen in eine Summe von Stammbrüchen mit paarweise Verschiedenen Nennern berechnen konnten. Die einfache Zerlegung
$$\frac{a}{b} = \underbrace{\uf{b} + \uf{b} + ... +\uf{b}}_{a-mal}$$
kam dabei nicht in Frage, da die Ägypter es als ,,unnatürlich'' ansahen, dass es mehr als diesen einen, wahren Teiler $\uf{b}$ einer Zahl geben sollte. \cite{Burton2011}\\
Für z.B. Brüche $\frac{2}{n}$ für $5 \leq n \leq 101$ ungerade findet sich dafür im Rhind Papyrus eine Tabelle mit der jeweiligen Zerlegung. Auch waren einige Regeln bekannt, beispielsweise
$$\frac{2}{n} = \uf{n} + \uf{2n} + \uf{3n} +\uf{6n}.$$
Tatsächlich wurde diese Regel in der zuvor genannten Tabelle des Papyrus nur einmal, bei $\frac{2}{101} = \uf{101}+\uf{202}+\uf{303}+\uf{606}$, verwendet, sonst wurden kürzere Zerlegungen gewählt. Trotz immenser Bemühungen ist es bisher nicht gelungen, das System zu ermitteln, mittels welchem diese Tabelle zustande kam. \cite{Burton2011}
\clearpage
\section{Algorithmen zur Erstellung Ägyptischer Brüche}\label{sec:algorithmen}
Im Folgenden sollen verschiedene Algorithmen erklärt und verglichen werden, mit welchen sich rationale Brüche in Ägyptische Brüche gemäß Definition \ref{def:egypfrac} zerlegen lassen. Die dabei aufgezeigten Methoden wurden über mehrere Jahrhunderte hinweg entwickelt und weisen dementsprechend signifikante Unterschiede auf. Im Anschluss an die Erklärung der Algorithmen soll ein Vergleich gezogen werden, der die Effizienz anhand der Kriterien ''Anzahl der verwendeten Stammbrüche'' und ''Länge des größten Nenners'' vergleicht.
Die nun betrachteten Algorithmen werden sein:
\begin{itemize}
	\item der Fibonacci-Sylvester-Algorithmus (auch: Greedy-Algorithmus)
	\item der Farey-Folgen-Algorithmus
	\item der Binär-Algorithmus.
\end{itemize}

Da im alten Ägypten noch keine negativen Zahlen bekannt waren, beschränken sich entsprechend die Algorithmen auf die positiven Dezimalbrüche.
\begin{def1}
	Da wir im Folgenden nur positive rationale Brüche zwischen 0 und 1 betrachten wollen, sei die Menge $\Q_+$ wie folgt definiert:
	$$\Q_+ := \left\{x \in \Q: 0<x<1\right\}.$$
\end{def1}
% !TeX spellcheck = de_DE_frami
% !TEX root = document.tex
\section{Der Greedy-Algorithmus}
Seien $a, b \in \N; b \neq 0$. Eine der bekanntesten Methoden, eine Ägyptische Erweiterungen für Brüche $\frac{a}{b}$ zu finden, ist der Greedy-Algorithmus. Dabei werden jeweils die größtmöglichen Stammbrüche $\frac{1}{x_i}$ gesucht, sodass
\begin{equation}\label{eq:greedy_fracNorm}
\frac{1}{x_i} \leq \frac{a}{b} - \sum_{j=1}^{i-1} \frac{1}{x_j} < \frac{1}{x_{i}-1},
\end{equation}
wobei gilt, dass
$$x_i \neq x_j; \forall i \neq j = (1,..,i)$$ 
\change[inline]{das ist nicht nachvollziehbar!}
solange, bis
$$\frac{a}{b} = \frac{1}{x_1} + \frac{1}{x_2} + ... + \frac{1}{x_i} = \sum_{j=1}^{i} \frac{1}{x_j}.$$
Da in jedem Fall der größtmögliche, noch nicht vorhandene Bruch gesucht wird, der noch in die Summe der Stammbrüche passt, ohne dass diese zu groß wird, kann es zu sehr ungünstigen Ergebnissen mit extrem langen Divisoren kommen; ein anschauliches Beispiel dafür ist:
$$\frac{5}{121} = \uf{25} + \uf{757} + \uf{763309} + \uf{873960180913} + \uf{1527612795642093418846225},$$
wobei man den Bruch auch folgendermaßen zerlegen kann:
$$\frac{5}{121} = \uf{33} + \uf{121} + \uf{363}.$$
Aufgrund dieser Komplexitätsprobleme scheint es unsinnig, den Greedy-Algorithmus zu verwenden. Nichtsdestotrotz lässt sich beweisen, dass der Greedy-Algorithmus immer terminiert.
Im Anhang \ref{code:greedy} findet sich eine eigene Implementierung des Greedy-Algorithmus.

\begin{satz}
	Der Greedy-Algorithmus, wie oben beschrieben, terminiert für jede Eingabe.
\end{satz}
\begin{bew}
	Für die erste Iteration des Greedy-Algorithmus ergibt sich aus \ref{eq:greedy_fracNorm}:
	\begin{equation*}
		\uf{x} \leq \frac{a}{b} < \uf{x-1}
	\end{equation*}
	Daraus folgt ein Rest $r$ von
	$$ r = \frac{a}{b}-\uf{x} = \frac{ax-b}{bx},$$
	der den Zähler $(ax-b)$ hat, \improvement{Beweis einfügen} der kleiner als a ist. \\ Somit verkleinert sich dieser Rest mit jedem Schritt und erreicht irgendwann Null, wie gefordert.
\end{bew}
\subsection{Der Farey-Folgen-Algorithmus}
Eine weitere Methode zur Erstellung Ägyptischer Brüche stellt der sogenannte Farey-Folgen-Algorithmus dar, der seinen Namen aus dem Umstand bezieht, dass die Farey-Folge dafür genutzt wird.
\begin{def1}
	Sei $q \in \N$. Die Farey-Folge der Ordnung $q$, $\, F_q$, ist definiert als die aufsteigend sortierte Folge aller einmalig darin vorkommenden gekürzten Brüche $\frac{a}{b} \in \Q$, für die gilt:
	$0\leq a \leq b \leq q,\, b\neq 0$.
\end{def1}

\begin{bsp}
	Sei $q=5$. Die Farey-Folge der Ordnung 5 ist
	$$F_5 = \left\{\frac{0}{1}, \uf{5}, \uf{4}, \uf{3}, \frac{2}{5}, \uf{2}, \frac{3}{5}, \frac{2}{3}, \frac{3}{4}, \frac{4}{5}, \uf{1} \right\}.$$
\end{bsp}

Der Algorithmus funktioniert dann wie folgt.
\begin{algorithm}\label{algo:FareySeries}
	Sei $\frac{p}{q} \in \Q_+$ in reduzierter Form der zu zerlegende Bruch.
	\begin{enumerate}
		\item Konstruiere $F_q$.
		\item Sei $\frac{r}{s}$ der zu $\frac{p}{q}$ adjazente Bruch in $F_q$, sodass $\frac{r}{s} < \frac{p}{q}$. Aufgrund der Eigenschaften der Farey-Folge gilt dann
		$$\frac{p}{q} = \uf{qs} + \frac{r}{s},$$ wobei $s<q,\, r<p$ \cite[S. 425]{Beck2000}.
		\item Wiederhole dieses Vorgehen für $\frac{r}{s}$ solange, bis $s=1 \gdw r=0$.
	\end{enumerate}
\end{algorithm}
\begin{satz}
	Algorithmus \ref{algo:FareySeries} terminiert für jede Eingabe.
\end{satz}
\begin{bew}
	Sei $\frac{p}{q} \in \Q_+$ der zu zerlegende rationale Bruch, $\frac{r}{s}, \frac{t}{u} \in \Q_+$ die zu $\frac{p}{q}$ in $F_q$ adjazenten Brüche, wobei gilt $\frac{r}{s}<\frac{p}{q} < \frac{t}{u}$.\\
	Es gilt $$\frac{p}{q} = \uf{qs}+\frac{r}{s}.$$
	Falls $r = 0$, dann $\frac{p}{q} = \uf{qs}$ und der Algorithmus terminiert.
	Sonst wird der Algorithmus für $\frac{r}{s}$ in $F_s$ wiederholt. Es gilt $s<q$, da kein Nenner in $F_q$ größer als $q$ ist und zwei beliebige Brüche mit demselben Nenner niemals adjazent zueinander sind. Somit wird nach endlich vielen Schritten $r=0$ erreicht. Da in jeder Farey-Folge der einzige Bruch mit Zähler Null $\frac{0}{1}$ ist, ist dies der Abbruchfall.
\end{bew}

\paragraph{Abschätzung}Sei $\frac{p}{q} \in \Q_+$ der zu zerlegende Bruch. Der Farey-Folgen-Algorithmus liefert dafür eine Zerlegung mit maximal $p$ Termen und einem größten Nenner von höchstens $q(q-1)$. \cite[S.343]{Bleicher1972}

Da die Farey-Folge $F_q$ schon für mäßig große $q$ sehr groß wird, bietet sich für das tatsächliche Berechnen eine Optimierung an, indem nur der relevante Teil der Farey-Folge konstruiert wird. Anwendung findet dabei das Bisektionsverfahren.
\begin{def1}\label{def:mediant}
	Die im Folgenden verwendete Mediante zweier rationaler Brüche $\frac{a}{b}, \frac{c}{d} \in \Q_+$ $mediant: \Q_+ \times \Q_+ \rightarrow \Q_+$ sei definiert als:
	$$mediant \left(\frac{a}{b}, \frac{c}{d}\right) = \frac{a+c}{b+d}.$$
\end{def1}
\begin{bsp}\label{bsp:Frel}
	Sei $\frac{p}{q} = \frac{21}{23}$. Die obere und untere Schranken sind in unserem Fall $0$ \bzw $1$. Die Mediante liegt also bei $\uf{2}$, wir stellen fest: $\uf{2}<\frac{21}{23}<1$, also setzen wir die Suche im Intervall $[\uf{2}, 1]$ fort. Die Mediante liegt nun bei $\frac{2}{3}$, $\frac{2}{3}<\frac{21}{23}<1$ \usw
	Bei der Mediante $\frac{11}{12}$ stellen wir fest: $\frac{10}{11} < \frac{21}{23} < \frac{11}{12}.$ Daraus folgt der relevante Teil von $F_{23}$, hier $F_{23rel}$ genannt: $$F_{23rel} = \left\{\frac{0}{1}, \frac{1}{2}, \frac{2}{3}, \frac{3}{4}, \frac{4}{5}, \frac{5}{6}, \frac{6}{7}, \frac{7}{8}, \frac{8}{9}, \frac{9}{10}, \frac{10}{11}, \frac{21}{23}, \frac{11}{12}\right\}.$$
	Weil für die Adjazenzberechnung von $\frac{p}{q}$ nur die unmittelbare Umgebung nötig ist, fallen alle Brüche, die größer als der größere der zu $\frac{p}{q}$ adjazenten Brüche sind, aus der gekürzten Farey-Folge heraus; im Beispiel betrifft dies $\uf{1}$.
	Somit enthält die gekürzte Farey-Folge nur noch 12 der andererseits 173 zu berechnenden Elemente von $F_{23}$.
\end{bsp}
\subsection{Binäralgorithmus}

Dass die Ägypter schon damals implizit eine Art Binärschreibweise für ihre Multiplikation und Division verwendeten, wie in Abschnitt \ref{sec:arithmetic} erläutert, ist inzwischen bekannt. Nun kann man den gleichen Umstand auch für einen Konstruktionsalgorithmus verwenden, wie im Folgenden gezeigt wird.\\
Seien $m, n \in \N$ und $N = 2^n$. Alle $m < N$ lassen sich als Summe paarweise verschiedener Teiler von $N$ beschreiben, folglich also mit maximal $n$ Summanden. Praktisch lässt sich dies am Einfachsten anhand der Binärdarstellung von $m$ erkennen.

\begin{algorithm}\label{algo:binary}
	Sei $\frac{p}{q} \in Q_+, \, \frac{p}{q} < 1$ in gekürzter Form und $k \in \N$.
	\begin{enumerate}
		\item Finde $N_{k-1} < q \leq N_k$ wobei $N_k=2^k$ ist.
		\item Falls $q = N_k$, schreibe $p$ als Summe von Teilern von $N_k$, hier $d_i$ genannt:
		$$\frac{p}{q} = \sum_{i=1}^{j} \frac{d_i}{N_k}=  \sum_{i = 1}^{j}\frac{1}{\frac{N_k}{d_i}}$$
		\item Sonst seien $s, r \in \N, 0 < r < N_k$ so gewählt, dass:
		$$pN_k = qs+r.$$
		Es folgt:
		$$\frac{p}{q} = \frac{p N_k}{q N_k} = \frac{qs + r}{q N_k} = \frac{s}{N_k} + \frac{r}{q N_k}$$.
		\item Schreibe $s = \sum d_i$ und $r = \sum d_i'$, wobei $d_i, d_i'$ jeweils paarweise verschiedene Teiler von $N_k$ sind.
		\item Erhalte den Ägyptischen Bruch:
		$$\sum \uf{\frac{N_k}{d_i}} + \sum \uf{\frac{q N_k}{d_i'}}$$
	\end{enumerate}
\end{algorithm}

\begin{satz}
	Der Binäralgorithmus, wie in \ref{algo:binary} beschrieben, terminiert für jede Eingabe.
\end{satz}

\begin{bew}
	Falls $q = N_k$, hat das Ergebnis offensichtlich maximal k Terme, da sich $N_k$ als Summe seiner $\log_2 N = k$ schreiben lässt; in der Summe sind die $d_i$ paarweise verschieden, somit auch die $\frac{N_k}{d_i}$ und es gibt keinen Term mehrfach.\\
	Falls $q < N_k$, gilt
	$$qs+r = p N_k < q N_k.$$
	Somit gibt es eine Zerlegung in Ägyptische Brüche jeweils für s und r. Diese beiden Zerlegungen liefern für sich genommen nach dem Argument aus Fall ''$q = N_k$'' paarweise verschiedene Stammbrüche. Dass diese sogar zusammengenommen paarweise verschieden sind, folgt daraus, dass die zu s gehörenden Nenner immer Zweierpotenzen sind, die zu r gehörigen aber niemals.
\end{bew}


Sei $\frac{p}{q} \in \Q_+, \frac{p}{q} < 1$. Gong zeigte, dass 
\todo[inline]{Komplexitätsabschätzung!}

Da es für diesen Algorithmus zwei Fälle gibt, soll für jeden Fall ein Beispiel gezeigt werden. Dafür beginnen wir mit dem einfachen Fall.

\begin{bsp}
	Sei $\frac{9}{16}$ der zu zerlegende Bruch. $N_k = 16$, da $8<16\leq16$.\\
	Da $16 = 16$, wird nach Schritt (2) aus Algorithmus \ref{algo:binary} verfahren:
	$$\frac{9}{16} = \frac{8+1}{16} = \uf{2}+\uf{16}.$$
\end{bsp}
Ist der Nenner des zu zerlegenden Bruchs also eine Zweierpotenz, terminiert der Algorithmus sehr schnell. Anders ist dies, sollte es sich um keine Zweierpotenz handeln, wie das nächste Beispiel zeigt.
\begin{bsp}
	Sei $\frac{21}{23}$ der zu zerlegende Bruch. $N_k = 32$, da $16 < 23 \leq 32$.\\
	Da $23<32$ ist, wird nach Schritt (3) aus Algorithmus \ref{algo:binary} verfahren:
	$$\frac{21}{23} = \frac{21 \cdot 32}{23 \cdot 32}.$$
	Aus der Bedingung $0<r<32$ folgt $s = 29$ und $r = 5$, da $$qs+r = 29 \cdot 23 + 5 = 21 \cdot 32 = p N_k.$$
	Daraus folgt:
	$$\frac{21}{23} = \frac{23 \cdot 29+5}{23 \cdot 32} = \frac{29}{32} + \frac{5}{23 \cdot 32}.$$
	Aus $$\frac{29}{32} = \frac{16+8+4+1}{32} = \uf{2} + \uf{4} + \uf{8} + \uf{32}$$ und $$\frac{5}{23 \cdot 32} = \frac{4+1}{23 \cdot 32} = \uf{23 \cdot 8} + \uf{23 \cdot 32}$$ folgt:
	$$\frac{21}{23} = \uf{2} + \uf{4} + \uf{8} + \uf{32} + \uf{184} + \uf{736}.$$
\end{bsp}
\subsection{Beispielrechnungen}

Anhand von drei Beispielen soll die Rechnung der einzelnen Methoden zur besseren Vergleichbarkeit verdeutlicht werden. Am Ende dieses Abschnitts sind die Ergebnisse tabellarisch gegenübergestellt.

\begin{bsp}
	Wir wollen zunächst einen einfachen Bruch zerlegen, für den alle drei der aufgezeigten Algorithmen das gleiche Ergebnis liefern, auch wenn sie dieses auf unterschiedlichem Wege erreichen.
	Sei $\frac{5}{9}$ der betrachtete Bruch.
	\paragraph{Greedy-Algorithmus} Es wird der größte Stammbruch $\uf{x_1} \in \Q_+$ gesucht mit $\uf{x_1} \leq \frac{5}{9} < \uf{x_1-1}$. Es folgt $$\uf{x_1} = \uf{2} \text{, da }
	\uf{2} \leq \frac{5}{9} < \uf{1}.$$
	Der verbleibende Rest $r$ ist dann $$r = \frac{5}{9} - \uf{2} = \uf{18}$$
	und der Algorithmus terminiert, da die Summe dieser beiden Stammbrüche genau $\frac{5}{9}$ entspricht.
	
	\paragraph{Farey-Folgen-Algorithmus}Wir bilden den notwendigen Teil der Farey-Folge mit Ordnung $9$: $$F_{9rel} = \left\{ \frac{0}{1}, \uf{2}, \frac{5}{9}, \frac{4}{7}, \frac{3}{5}, \frac{2}{3}, \uf{1}\right\},$$ woraus folgt, dass $\uf{2}$ der zu $\frac{5}{9}$ adjazente Bruch in $F_9$ ist. Daraus folgt 
	$$\frac{5}{9} = \uf{9 \cdot 2} + \uf{2} = \uf{2} + \uf{18}$$ und der Algorithmus terminiert nach dem ersten Schritt, da der Rest $\frac{r}{s} = \uf{2}$ schon Stammbruch ist.
	
	\paragraph{Binäralgorithmus}Die kleinste Zweierpotenz, die größer als der Nenner $9$ ist, ist $16$, da $8<9<16$; somit ist $N_k = 16$. Da der Nenner keine Zweierpotenz ist, springen wir zu Schritt 3 des Algorithmus \ref{algo:binary} und schreiben:
	$$\frac{5}{9} = \frac{5\cdot16}{9 \cdot 16} = \frac{9 \cdot 8 + 8}{9 \cdot 16} = \frac{8}{16} + \frac{8}{144} = \uf{2} + \uf{18}.$$
	
	Es ergibt sich also bei allen Algorithmen das gleiche Ergebnis, obwohl die Herangehensweise sehr unterschiedlich ist. Das ist aber nur ein Ausnahmefall, die folgenden Beispiele werden zeigen, wie unterschiedlich die Ergebnisse werden können.
	
	\vspace{0.5cm}
	\begin{table}[H]
		\centering
		\begin{tabular}{|Sc | Sc | Sc | Sc|}
			\hline
			Algorithmus & Anzahl der Terme & Größter Nenner & Zerlegung \\ \hline
			Greedy & 2 & 18 & $\uf{2}+\uf{18}$ \\ \hline
			Farey-Folgen-Algorithmus & 2 & 18 & $\uf{2}+\uf{18}$ \\ \hline
			Binär-Algorithmus & 2 & 18 & $\uf{2}+\uf{18}$ \\ \hline
		\end{tabular}
		\caption{Die Zerlegung von $\frac{5}{9}$ im Vergleich}
		\label{table:vgl5/9}
	\end{table}
\end{bsp}


\vspace{2cm}
\begin{bsp}
	Sei $\frac{24}{31}$ der zu zerlegende Bruch. Der Einfachheit halber werden hier die Lösungswege nur noch angeschnitten.
	
	\paragraph{Greedy-Algorithmus} Wieder suchen wir iterativ die größten Stammbrüche, bis diese in ihrer Summe $\frac{24}{31}$ ergeben. Nach 4 Schritten erhält man das Ergebnis:
	$$\frac{24}{31} = \uf{2} + \uf{4} + \uf{42} + \uf{2604}.$$
	
	\paragraph{Farey-Folgen-Algorithmus}Wie bisher wird zunächst der zu $\frac{24}{31}$ in $F_{31}$ adjazente Bruch gesucht, mit dem nach Rechenvorschrift fortgefahren wird, sodass sich nach der ersten Iteration
	$$\frac{24}{31}=\uf{31 \cdot 22}+ \frac{17}{22} = \uf{682} + \frac{17}{22}$$
	ergibt. Mit den folgenden 5 Iterationen ergibt sich:
	$$\frac{24}{31}=\uf{2}+\uf{6}+\uf{12}+\uf{52}+\uf{286}+\uf{682}.$$
	
	\paragraph{Binäralgorithmus}Schritt 1 liefert $N_k = 32$. Schritt 3 zufolge gilt
	$$\frac{24}{31} = \frac{24 \cdot 31 + 24}{31\cdot32} = \frac{16+8}{32} + \frac{16+8}{31 \cdot 32} = \uf{2} + \uf{4} + \uf{62} + \uf{124}.$$
	
	Die Ergebnisse, die in Tabelle \ref{table:vgl24/31} aufgelistet sind, zeigen einige signifikante Unterschiede: Vergleicht man den Greedy- mit dem Farey-Folgen-Algorithmus,  wird deutlich, dass durch die Erhöhung der Anzahl der Terme von 4 auf 6 der größte vorkommende Nenner um mehr als den Faktor 3 verringert werden kann. Es geht aber offensichtlich noch besser, denn der Binäralgorithmus liefert nur 4 Terme, deren größter Nenner aber um den Faktor 21 kleiner ist als der des Greedy-Algorithmus und um den Faktor 5,5 kleiner als der des Farey-Folgen-Algorithmus. Ersteres ist allein der Habgier\footnote{greedy, engl. für: habgierig, gierig, gefräßig} des Greedy-Algorithmus geschuldet, da er an dritter Stelle statt der besseren Wahl $\uf{62}$ den größeren Bruch $\uf{42}$ wählt und somit den Rest so stark verkleinert, dass dieser nur durch einen relativ großen Nenner ausgedrückt werden kann. Hier ist also der Binäralgorithmus den anderen beiden deutlich überlegen.

	\vspace{0.5cm}
	\begin{table}[H]
		\centering
		\begin{tabular}{|Sc | Sc | Sc | Sl|}
			\hline
			Algorithmus & Anzahl der Terme & Größter Nenner & Zerlegung \\ \hline
			Greedy & 4 & 2604 & $\uf{2} + \uf{4} + \uf{42} + \uf{2604}$ \\ \hline
			Farey-Folgen-Algorithmus & 6 & 682 & $\uf{2}+\uf{6}+\uf{12}+\uf{52}+\uf{286}+\uf{682}$ \\ \hline
			Binär-Algorithmus & 4 & 124 & $\uf{2} + \uf{4} + \uf{62} + \uf{124}$ \\ \hline
		\end{tabular}
		\caption{Die Zerlegung von $\frac{24}{31}$ im Vergleich}
		\label{table:vgl24/31}
	\end{table}
\end{bsp}
\vspace{2cm}
\begin{bsp}
	Dass der Greedy-Algorithmus nicht grundsätzlich der schlechteste ist, zeigt das Beispiel $\frac{12}{17}$.
	\paragraph{Der Greedy-Algorithmus} liefert
	$$\frac{12}{17} = \uf{2}+\uf{5}+\uf{170}.$$
	\paragraph{Der Farey-Folgen-Algorithmus} liefert
	$$\frac{12}{17}=\uf{2}+\uf{6}+\uf{30}+\uf{170}.$$
	\paragraph{Der Binäralgorithmus} liefert
	$$\frac{12}{17} = \uf{2}+\uf{8}+\uf{16}+\uf{68}+\uf{272}.$$
	Wie die Zusammenfassung in Tabelle \ref{table:vgl12/17} zeigt, ist in diesem Beispiel der Greedy-Algorithmus leicht überlegen. An den Zweierpotenzen der ersten Terme des Binäralgorithmus lässt sich sehr gut dessen Natur kennen, die ihm aber in diesem Beispiel nicht zu sonderlicher Effizienz verhilft.

	\vspace{0.5cm}
	\begin{table}[H]
		\centering
		\begin{tabular}{|Sc | Sc | Sc | Sl|}
			\hline
			Algorithmus & Anzahl der Terme & Größter Nenner & Zerlegung \\ \hline
			Greedy & 3 & 170 & $\uf{2}+\uf{5}+\uf{170}$ \\ \hline
			Farey-Folgen-Algorithmus & 4 & 170 & $\uf{2}+\uf{6}+\uf{30}+\uf{170}$ \\ \hline
			Binär-Algorithmus & 5 & 272 & $\uf{2}+\uf{8}+\uf{16}+\uf{68}+\uf{272}$ \\ \hline
		\end{tabular}
		\caption{Die Zerlegung von $\frac{12}{17}$ im Vergleich}
		\label{table:vgl12/17}
	\end{table}
\end{bsp}


Es zeigt sich also, dass sich anhand relativ weniger Beispiele nicht sagen lässt, ob ein Algorithmus besser ist als ein anderer. Dazu braucht es einen großen, systematisch aufgebauten Datensatz, der dann statistisch ausgewertet wird. Trotzdem lässt sich daraus nicht ableiten, welcher Algorithmus im Einzelfall besser funktioniert. Viele Algorithmen sind mit aufwändigen Beweisen ihrer Schranken bezüglich ''Länge des größten Nenners'' und ''Anzahl der entstehenden Terme'' verbunden, diese sagen aber selten etwas über die durchschnittlichen Ergebnisse aus.
\subsection{Fazit und Vergleich}
Wie in den vorangegangenen Beispielen zu sehen, liefern verschiedene Methoden z.T. stark voneinander abweichende Ergebnisse. Wie optimal zwei Ergebnisse im Vergleich zueinander sind, wird oft in den Maßeinheiten der oberen Schranken der ''Länge des größten Nenners'' \bzw der ''Anzahl der Summanden'' angegeben. Sei $\frac{p}{q} \in \Q_+$ der untersuchte Bruch. In Tabelle \ref{table:Algo_Vergleich} sind die entsprechenden Werte zum Vergleich aufgelistet.\cite[S. 343]{Bleicher1972}
\vspace{0.5cm}
\begin{table}[H]
	\centering
	\begin{longtable}{|Sc | Sc | Sc|}
		\hline
		Algorithmus & Anzahl Summanden & Größtmöglicher Nenner\\
		\hline
		Fibonacci-Sylvester\newline(Greedy) & $p$ & - \\
		\hline
		Farey-Folgen-Algorithmus & $p$ & $q^2-q$\\
		\hline
		Binäralgorithmus & $O(\log q)$ & $2(q^2-q)$\\
		\hline
	\end{longtable}
	\caption{Vergleich der beschriebenen Algorithmen (obere Schranken)}
	\label{table:Algo_Vergleich}
\end{table}
Für den größtmöglichen Nenner des Greedy-Algorithmus gibt es eine Besonderheit, denn die Nenner dort wachsen oft exponentiell, weshalb sich keine brauchbare obere Schranke angeben lässt \cite[S. 157]{BleicherErdoes1976}.
\clearpage
\section{Auswertung einiger Testreihen}\label{sec:Testreihen}
Zur effizienten Untersuchung und Rechnung zahlreicher Beispiele wurden die in Kapitel \ref{sec:algorithmen} aufgezeigten Algorithmen in PARI/GP umgesetzt. (\cite{PARI2018})\\
Alle Beispiele wurden dabei ausschließlich mit dem in Abschnitt \ref{sec:code} gelisteten Programmcode gerechnet.

\subsection{Methodik}
Zur Testung der genannten Algorithmen wurde die folgende Methodik umgesetzt.
Alle drei Algorithmen durchliefen die Tests zwecks Vergleichbarkeit auf sehr ähnlichen Rechnern mit gleichen Bedingungen, die Laufzeiten sind somit vergleichbar.
Getestet wurde ausgehend von $n \in \N$ mit $3 \leq n \leq 10.000$.
Der Datensatz für ein gewähltes $n$ definiert sich durch die Menge der betrachteten Brüche $M_n = \left\{ \frac{x}{n} \, | \, 1\leq x < n \wedge x \nmid n\right\}$ und enthält neben dem $n$ selbst folgende Eigenschaften, die zugleich den nachgestellten Namen bekommen:
\begin{itemize}
	\item die durchschnittliche Anzahl der Summanden, \emph{avgTerms(n)}
	\item das Minimum der Anzahl der Summanden, \emph{minTerms(n)}
	\item das Maximum der Anzahl der Summanden, \emph{maxTerms(n)}
	\item das Minimum des jeweils größten Nenners, \emph{minDenom(n)}
	\item das Maximum des jeweils größten Nenners, \emph{maxDenom(n)}.
\end{itemize}
Somit entstanden 10.000 Datensätze pro Algorithmus mit jeweils 5 nutzbaren Parametern, insgesamt also 150.000 Datenpunkte, die im Folgenden ausgewertet werden sollen.

\subsection{Anfängliche Probleme und Effizienzsteigerung während der Implementierung}
In den ersten Testläufen traten einige Unannehmlichkeiten auf, da Teile des Codes in der direkten Umsetzung der formalen Beschreibung nicht immer optimal liefen.
\paragraph{Suche nach Nennern im Greedy-Algorithmus}Beispielsweise gab es im ursprünglichen Test des Greedy-Algorithmus den Fall der Zerlegung von $\frac{5}{121}$, dessen Berechnung mittels der Funktion \emph{greedy(5/121)} nach ca. 7 Stunden ohne Ergebnis abgebrochen wurde. Grund hierfür ist das Problem, den größten Stammbruch $\uf{x} \in \Q_+$ zu finden, der kleiner als ein gegebener rationaler Bruch $\frac{p}{q} \in \Q_+$ ist. In der ersten Implementierung wurde dafür in der Folge $(\uf{2}, \uf{3}, \uf{4}, \uf{5},...,\uf{x})$ gesucht, was für große Nenner offensichtlich sehr lange dauert. Nachdem eine Optimierung vorgenommen wurde, in der nun zunächst die Nenner der genannten Folge verdoppelt und dann beim Halbieren aufsummiert werden $(\uf{2}, \uf{4}, \uf{8},...,\uf{n_i}, \uf{n_i+n_j}, \uf{n_i+n_j+n_k},...,\uf{x}), \text{wobei alle } n \text{ Zweierpotenzen sind und } n_i>n_j>n_k > ... \geq 1; \, i,j,k \in \N$, steigerte sich die Effizienz des Greedy-Algorithmus enorm: Das genannte Beispiel $\frac{5}{121}$ wurde in unter einer Millisekunde korrekt berechnet, zahlreiche weitere Beispiele zeigten ähnliche Auswirkungen. Deshalb trägt diese Implementierung auch den Namen \emph{greedy\_fast(fraction)}.

\paragraph{Konstruktion der Farey-Folgen}Ein weiteres Problem stellte die Berechnung der Farey-Folgen dar. Einerseits haben solche Folgen $F_q$ selbst mit mäßig großem $q$ schon sehr viele Elemente, so ist \zB $|F_{280}| = 23.861$. Aus den Eigenschaften der Farey-Folgen kann man sehen, dass diese mit steigender Ordnung exponentiell an Größe gewinnen. Andererseits werden diese Elemente mit steigender Ordnung auch größer, was also zusätzlich Speicher belegt und Rechenzeit beansprucht. Nach den ersten paar Rechnungen wurde schnell klar, dass die vollständige Berechnung der Farey-Folgen nicht ansatzweise optimal ist und deshalb der Ansatz der Berechnung ausschließlich des relevanten Teils der Farey-Folge, wie es in Beispiel \ref{bsp:Frel} beschrieben ist, umgesetzt wurde. Dies brachte eine signifikante Effizienzsteigerung mit sich.

\subsection{Auswertung der Daten}

\subsubsection{Laufzeiten}
Die Algorithmen erreichten die nachfolgend aufgeführten Laufzeiten für die Datensätze $3 \leq n \leq 10.000$:\\
\begin{table}[H]
	\centering
	\begin{tabular}{|l | c c c c|}
		\hline
		\multicolumn{1}{|c|}{\emph{Algorithmus}} & h & min & sek & ms \\ \hline
		Binär-Algorithmus & & 33 & 22 & 336 \\ \hline
		Greedy-Algorithmus & & 40 &  53 & 499 \\ \hline
		Farey-Folgen-Algorithmus & 62 & 33 & 38 & 136 \\ \hline
	\end{tabular}
	\caption{Laufzeitvergleich der Algorithmen}
	\label{table:LaufzeitVgl}
\end{table}
Die starke Unterlegenheit des Farey-Folgen-Algorithmus ist klar erkenntlich, allerdings nicht mehr wesentlich zu verbessern. Dieses Ergebnis entstand schon mit der in Beispiel \ref{bsp:Frel} beschriebenen Methode, nur die relevante Farey-Folge zu konstruieren, da die Laufzeit mit vollständiger Konstruktion der jeweiligen Farey-Folge für jeden Datensatz $n$ noch höher war. Es lässt sich folgern, dass er damit der laufzeittechnisch schlechteste der drei Algorithmen ist.

\subsubsection{Durchschnittliche Anzahl der Terme}
Eines der wichtigsten Ergebnisse zur realistischen Abschätzung der Algorithmen gegeneinander stellt die durchschnittliche Anzahl der Terme dar, die jeweils durch die Zerlegung entstehen. Abbildung zeigt graphisch die Entwicklung. Es ist klar zu erkennen, wie sich Greedy- und Binary-Algorithmus früh bei einem relativ festen Wert mit konstanten Schwankungen einpendeln, der Ausschlag der Kurve für den Farey-Folgen-Algorithmus aber signifikant stärker ist und sich erst später bei einem wesentlich höheren Wert einpendelt. Alle Algorithmen haben aber starke Schwankungen, es gibt also keinen festen Wert, der in bestimmten Bereichen eingenommen wird.
\todo[inline]{insert graphics}

\subsubsection{Minimale Anzahl der Terme}
Bezüglich der minimalen Anzahl der produzierten Terme liefert der Greedy-Algorithmus für alle $n$ den gleichen Wert $minTerms_{greedy}(n) = 2$, Der Farey-Folgen-Algorithmus liefert ständig zwischen $2$ und $3$ alternierende Werte, wobei $n=6$ eine Ausnahme liefert mit $minTerms_{farey}(6) = 5$, da aufgrund der vielen Kürzungen im Datensatz für $n=6$ nur $\frac{5}{6}$ betrachtet wird, welcher in der Farey-Zerlegung 5 Terme benötigt. Beim Binär-Algorithmus erreichen die Meisten Tests ein Minimum von 2 Termen pro Datensatz, in ca. $0,7\%$ Fälle liegt das Minimum jedoch bei 3 Termen. Ein besonderer Zusammenhang zwischen solchen Datensätzen wurde nicht gefunden.
\todo[inline]{insert graphics}

\subsubsection{Maximale Anzahl der Terme}
 Der Greedy-Algorithmus pendelt sich ab $n>3000$ bei Werten von $7 \leq maxTerms_{greedy}(n) \leq 16$ ein, während der Binäralgorithmus sich ab $n>2000$ mit Werten von $15 \leq maxDenom_{binary}(n) \leq 22$ einpendelt, wobei einzelne Abweichungen nach oben und unten vernachlässigt wurden. Für den Farey-Folgen-Algorithmus hingegen gilt ein lineares Wachstum:
$$maxTerms_{farey}(n) = n + 1, \, \forall n$$
für \todo{finde Erklärung}alle getesteten Datensätze. 
\todo[inline]{insert graphics}

\subsubsection{Minimum der größten Nenner}
In dieser Hinsicht zeichnet sich ein besonders interessantes Bild ab. Alle Algorithmen zeigen klares lineares Wachstum bezüglich $minDenom(n)$, allerdings in sehr unterschiedlichen Ausprägungen. Der Binäralgorithmus bewegt sich zwischen den Werten der anderen Algorithmen und ist, bis auf wenige Ausnahmen, durch eine einzige Gerade erkenntlich. Der Greedy-Algorithmus, dessen Werte dieses Kriteriums unterhalb derer des Binäralgorithmus liegen, zeigt hingegen mehrere Geraden auf, also entwickeln sich bestimmte Folgen innerhalb der Testreihe dieses Algorithmus gleich, die Folgen sind aber in ihrer Entwicklung voneinander unterscheidbar. Gleiches gilt für den mit größeren Werten ergebenden Farey-Folgen-Algorithmus, für den sich 5 geraden ergeben, allerdings wird hier die Anzahl der Datenpunkte, die jeweils eine Gerade ergeben, mit wachsendem Anstieg selbiger, immer weniger. \emph{Es ist zu vermuten, dass dies von gewissen Modulae abhängt, weitere Erkenntnisse fehlen noch}
\todo[inline]{insert graphics}
\subsubsection{Maximum der größten Nenner}
Dieses Kriterium ist jenes, in welchem sich die Algorithmen wohl am stärksten voneinander unterscheiden. Während sich die Werte von Binär- und Farey-Folgen-Algorithmus unter $10^{19}$ halten, streuen sich die Werte des Greedy-Algorithmus von $maxDenom_{greed}(3) = 2$ bis zu einem Maximum von $maxDenom_{greedy}(4967) \approx 7,3378 \times 10^{225.516}$ ohne jegliche erkennbare Regelmäßigkeit. \emph{binary \& farey relativ regelmäßig (Vgl. quadratisch o.ä.?)}
\todo[inline]{insert graphics}
\subsubsection{Zusammenhang innerhalb des Binary-Algorithmus}
Im Binary-Algorithmus konnte eine unerwartete Korrelation zwischen $n$ und $minDenom(n)$ gemacht werden:
\begin{equation*}
	minDenom(n) = 
	\begin{cases}
		n & \text{falls n Zweierpotenz ist} \\
		2n & \text{sonst.}
	\end{cases}
\end{equation*}

\subsection{Zusammenfassung}
\begin{itemize}
	\item greedy zwar der beste bzgl. \#Terms, aber mit maxDenom fast unbrauchbar $\rightarrow$ sehr unvorhersehbar/ unkontrollierbar
	\item Farey liefert zwar brauchbare Größenordnung für die Länge der Nenner, aber dafür extrem viele Terme
	\item Binär ist wohl der am stabilsten laufende und brauchbarste, zudem der schnellste der getesteten Verfahren.
\end{itemize}
\clearpage
\section{Theoretische Grenzen}\label{sec:theorie}
Nachdem untersucht wurde, was in der Praxis mit ausgewählten Methoden erreicht werden kann, soll nun das Licht auf einen weiteren sehr interessanten Aspekt Ägyptischer Brüche gelenkt werden, nämlich den Dingen, die theoretische Grenzen haben. Vor Allem im 20. Jahrhundert wurde von zahlreichen Mathematikern an dem Problem, solche theoretischen Grenzen zu finden und zu beweisen, gearbeitet, unter ihnen Bleicher, Erdös, Graham und andere \cite[S.87 ff]{Guy1981}.

\subsection{Maximal benötigte Anzahl der Stammbrüche}
\begin{satz}\label{satz:two/n}
	Sei $n \in \N$ ungerade. $\frac{2}{n}$ lässt sich für jedes $n$ als Summe zweier Stammbrüche notieren, nämlich:
	$$\frac{2}{n} = \uf{\lceil \frac{n}{2} \rceil} + \uf{n \cdot \lceil \frac{n}{2} \rceil}.$$
\end{satz}
\begin{bew}
	Sei $m \in \N$ so gewählt, dass $n=2m+1$. Es folgt:
	\begin{eqnarray*}
		\uf{\lceil \frac{n}{2} \rceil} + \uf{n \cdot \lceil \frac{n}{2} \rceil} & = & \uf{m+1} + \uf{(2m+1)(m+1)}\\
		& = & \frac{2m+1}{(2m+1)(m+1)} + \frac{1}{(2m+1)(m+1)}\\
		& = & \frac{2m+2}{(2m+1)(m+1)}\\
		& = & \frac{2(m+1)}{(2m+1)(m+1)}\\
		& = & \frac{2}{2m+1}\\
		& = & \frac{2}{n}.
	\end{eqnarray*}
\end{bew}

Da der Term $\uf{\lceil \frac{n}{2} \rceil}$ genau dem größten Stammbruch entspricht, der kleiner als $\frac{2}{n}$ ist, liefert der Greedy-Algorithmus hier für alle $\frac{2}{n}$ das gleiche Ergebnis wie die Rechenvorschrift aus Satz \ref{satz:two/n}. Damit lässt sich auch sofort eine weiterer Satz aufstellen, diesmal für $\frac{3}{n}$.

\begin{satz}
	Für jedes beliebige $n \in \N$ lässt sich $\frac{3}{n}$ als Summe von höchstens 3 Stammbrüchen schreiben.
\end{satz}
\begin{bew}
	Es gilt
	$$\frac{3}{n} = \uf{n} + \frac{2}{n}.$$
	Falls $n$ gerade ist, reichen sogar nur zwei Stammbrüche, denn dann kann $\frac{2}{n} = \uf{\frac{n}{2}}$ geschrieben werden; anderenfalls wird $\frac{2}{n}$ wie nach Satz \ref{satz:two/n} zerlegt und es ergeben sich genau drei Stammbrüche:
	$$\frac{3}{n} = \uf{n} + \uf{\lceil \frac{n}{2} \rceil} + \uf{n \cdot \lceil \frac{n}{2} \rceil}.$$
\end{bew}
\clearpage
% !TEX root = ../document.tex
\section{Anhang}
\subsection{PARI/GP Code für den Greedy-Algorithmus}\label{code:greedy}
Die Umsetzung aller Code-Beispiele erfolgte in PARI/GP. (\cite{PARI2018})
\lstinputlisting[language=gp, firstline=1, lastline=37]{../PARI/algorithms.gp}


	
	
	
% ######################################
% #                                    #
% #        LITERATURVERZEICHNIS        #
% #                                    #
% ######################################
	\clearpage
	\renewcommand{\refname}{Literaturverzeichnis}
	\bibliography{Bibliography}
	%\bibliographystyle{plain}
	\bibliographystyle{TeX-StyleDefinition/natdin}

% ######################################
% #                                    #
% #             ERKLÄRUNG              #
% #                                    #
% ######################################
	% !TEX root = dokument.tex
\clearpage
\addsec{Eidesstattliche Erklärung}

% Hinweis: die eidesstattliche Erklärung ist ggfs. an die Richtlinie der IHK anzupassen

Hiermit versichere ich, dass die vorliegende Arbeit selbständig verfasst und keine anderen als die angegebenen Quellen und Hilfsmittel benutzt wurden.\\Ferner habe ich vom Merkblatt über die Verwendung von Bachelor/Masterabschlussarbeiten Kenntnis genommen und räume das einfache Nutzungsrecht an meiner Bachelorarbeit der Universität der Bundeswehr München ein.\\[6ex]

\abgabeOrt, den \abgabeTermin


\rule[-0.2cm]{5.5cm}{0.5pt}

\textsc{\autorName}

	
	
\end{document}